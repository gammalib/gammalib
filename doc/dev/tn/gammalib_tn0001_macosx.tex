%%%%%%%%%%%%%%%%%%%%%%%%%%%%%%%%%%%%%%%%%%%%%%%%%%%
% Mac OS X deployment
%%%%%%%%%%%%%%%%%%%%%%%%%%%%%%%%%%%%%%%%%%%%%%%%%%%

%%%%%%%%%%%%%%%%%%%%%%%%%%%%%%%%%%%%%%%%%%%%%%%%%%%
% Definitions for manual package
%%%%%%%%%%%%%%%%%%%%%%%%%%%%%%%%%%%%%%%%%%%%%%%%%%%
\newcommand{\task}{\mbox{GammaLib}}
\newcommand{\this}{\mbox{\tt \task}}
\newcommand{\shorttype}{\mbox{}}
\newcommand{\doctype}{\mbox{Mac OS X deployment}}
\newcommand{\reference}{\mbox{GammaLib-TN0001}}
\newcommand{\version}{\mbox{1.0}}
\newcommand{\calendar}{\mbox{24 October 2011}}
\newcommand{\auth}{\mbox{J\"urgen Kn\"odlseder}}
\newcommand{\approv}{\mbox{J\"urgen Kn\"odlseder}}


%%%%%%%%%%%%%%%%%%%%%%%%%%%%%%%%%%%%%%%%%%%%%%%%%%%
% Document definition
%%%%%%%%%%%%%%%%%%%%%%%%%%%%%%%%%%%%%%%%%%%%%%%%%%%
\documentclass{article}[12pt,a4]
\usepackage{epsfig}
\usepackage{technote}


%%%%%%%%%%%%%%%%%%%%%%%%%%%%%%%%%%%%%%%%%%%%%%%%%%%
% Begin of document body
%%%%%%%%%%%%%%%%%%%%%%%%%%%%%%%%%%%%%%%%%%%%%%%%%%%
\begin{document}
\frontpage


%%%%%%%%%%%%%%%%%%%%%%%%%%%%%%%%%%%%%%%%%%%%%%%%%%%
% Scope
%%%%%%%%%%%%%%%%%%%%%%%%%%%%%%%%%%%%%%%%%%%%%%%%%%%
\section{Scope}

This technical note collects information about deployment of \this\ on Mac OS X, with
special emphasis on compatibility issues between different Mac OS X versions and 
CPU architectures.


%%%%%%%%%%%%%%%%%%%%%%%%%%%%%%%%%%%%%%%%%%%%%%%%%%%
% Introduction
%%%%%%%%%%%%%%%%%%%%%%%%%%%%%%%%%%%%%%%%%%%%%%%%%%%
\section{Introduction}

The Mac OS X operating systems comes in a number of versions and architectures which are
summarized in Tables \ref{table:versions} and \ref{table:arch}.
In particular, changes from Mac OS X 10.3 to 10.7 reflect the evolution from a pure PowerPC
system (10.3) to a pure Intel system (10.7).
To assure portability, \task\ should build and work properly on all these variants.

%%% Mac OS X versions %%%%%%%%%%%%%%%%%%%%%%%%%%%%%%%%%%%%%%
\begin{table}[!h]
  \center
  \begin{tabular}{lll}
  \hline
  Version & Name & New features \\
  \hline
  10.0 & Cheetah & \\
  10.1 & Puma & \\
  10.2 & Jaguar & \\
  10.3 & Panther & last PowerPC only system \\
  10.4 & Tiger & first Intel processor support, supports universal binaries (only 32-bit) \\
  10.5 & Leopard & first Intel 64-bit universal binary support \\
  10.6 & Snow Leopard & includes 32-bit and 64-bit kernels \\
  10.7 & Lion & PowerPC support dropped \\
  \hline
  \end{tabular}
  \caption{Mac OS X versions}
  \label{table:versions}
\end{table}
%%%%%%%%%%%%%%%%%%%%%%%%%%%%%%%%%%%%%%%%%%%%%%%%%%%

%%% Mac OS X architectures %%%%%%%%%%%%%%%%%%%%%%%%%%%%%%%%%%%%
\begin{table}[!h]
  \center
  \begin{tabular}{lll}
  \hline
  Architecture & Bit & Processor \\
  \hline
  {\tt ppc} & 32 & PowerPC \\
  {\tt i386} & 32 & Intel \\
  {\tt ppc64} & 64 & PowerPC \\
  {\tt x86\_64} & 64 & Intel \\
  \hline
  \end{tabular}
  \caption{Mac OS X architectures}
  \label{table:arch}
\end{table}
%%%%%%%%%%%%%%%%%%%%%%%%%%%%%%%%%%%%%%%%%%%%%%%%%%%

Multi-architecture support is implemented in Mac OS X in form of universal binaires (or libraries).
Here, code compiled for different architectures is packed into a single binary of library.
Using the {\tt gcc} compiler suite, this feature is automatically enabled when different
architectures are specified using the {\tt -arch} option to the compiler.
For example, {\tt -arch i386 -arch ppc} will built 32-bit code for both PowerPC and Intel
architectures.

To see which architectures exist in a binary or library, one can use the {\tt file} command.
For example
{\scriptsize
\begin{verbatim} 
$ file /usr/lib/libreadline.dylib 
/usr/lib/libreadline.dylib: Mach-O universal binary with 3 architectures
/usr/lib/libreadline.dylib (for architecture x86_64):	Mach-O 64-bit dynamically linked shared library x86_64
/usr/lib/libreadline.dylib (for architecture i386):	Mach-O dynamically linked shared library i386
/usr/lib/libreadline.dylib (for architecture ppc7400):	Mach-O dynamically linked shared library ppc
\end{verbatim}}
shows that the shared {\tt readline} library on the Mac OS 10.6 system is a 3-way library that
was compiled for {\tt x86\_64}, {\tt i386}, and {\tt ppc} architectures.

As the Mac OS X system evolves, new system features become available.
In order to compile code for older systems on a new Mac OS X version, Apple's Xcode comes
with a number of Software Development Kits (SDKs).
They are available in the {\tt /Developer/SDKs} folder.
Using these SDKs allows for building against the headers and libraries of an operating
system version other than the development platform.
This is achieved by adding the {\tt -isysroot} to the {\tt gcc} call.
For example, {\tt -isysroot /Developer/SDKs/MacOSX10.4u.sdk} builds and links against
headers and libraries that were available in Mac OS 10.4.
By selecting a given SDK, the build can use features available in OS versions up to and including 
the one corresponding to the base SDK.

To illustrate this, we can again use the {\tt file} command to see for what architectures the {\tt readline}
library was built in different Mac OS X versions:
{\scriptsize
\begin{verbatim} 
$ file /Developer/SDKs/MacOSX10.4u.sdk/usr/lib/libreadline.dylib
/Developer/SDKs/MacOSX10.4u.sdk/usr/lib/libreadline.dylib: Mach-O universal binary with 2 architectures
/Developer/SDKs/MacOSX10.4u.sdk/usr/lib/libreadline.dylib (for architecture i386):	Mach-O dynamically linked shared library stub i386
/Developer/SDKs/MacOSX10.4u.sdk/usr/lib/libreadline.dylib (for architecture ppc):	Mach-O dynamically linked shared library stub ppc
$ file /Developer/SDKs/MacOSX10.5.sdk/usr/lib/libreadline.dylib
/Developer/SDKs/MacOSX10.5.sdk/usr/lib/libreadline.dylib: Mach-O universal binary with 4 architectures
/Developer/SDKs/MacOSX10.5.sdk/usr/lib/libreadline.dylib (for architecture ppc7400):	Mach-O dynamically linked shared library stub ppc
/Developer/SDKs/MacOSX10.5.sdk/usr/lib/libreadline.dylib (for architecture ppc64):	Mach-O 64-bit dynamically linked shared library stub ppc64
/Developer/SDKs/MacOSX10.5.sdk/usr/lib/libreadline.dylib (for architecture i386):	Mach-O dynamically linked shared library stub i386
/Developer/SDKs/MacOSX10.5.sdk/usr/lib/libreadline.dylib (for architecture x86_64):	Mach-O 64-bit dynamically linked shared library stub x86_64
$ file /Developer/SDKs/MacOSX10.6.sdk/usr/lib/libreadline.dylib
/Developer/SDKs/MacOSX10.6.sdk/usr/lib/libreadline.dylib: Mach-O universal binary with 3 architectures
/Developer/SDKs/MacOSX10.6.sdk/usr/lib/libreadline.dylib (for architecture x86_64):	Mach-O 64-bit dynamically linked shared library stub x86_64
/Developer/SDKs/MacOSX10.6.sdk/usr/lib/libreadline.dylib (for architecture i386):	Mach-O dynamically linked shared library stub i386
/Developer/SDKs/MacOSX10.6.sdk/usr/lib/libreadline.dylib (for architecture ppc7400):	Mach-O dynamically linked shared library stub ppc
\end{verbatim}}
We note that on Mac OS X 10.4, {\tt readline} was compiled for {\tt i386} and {\tt ppc} only, while on
10.5 it was a 4-way library supporting PowerPC and Intel 32-and 64-bit architectures.
In Mac OS X 10.6, {\tt ppc64} support was dropped.
Note also that the libraries included in the SDK are stubs for linking only.
Stubs means that they do not contain executable code but just the exported symbols. 

Another important feature in compiling and running universal binaries (or libraries) is the 
Deployment Target (DT).
The Deployment Target specifies the earliest OS version on which the software can run.
It is important to recognize that the Deployment Target can be smaller than the SDK version used
for the build.
For example, one may specify a Deployment Target of 10.3 while using a SDK of 10.6.
This means that the code can exploit a feature that is only available in SDK 10.6, but it has 
to make sure itself that this feature is indeed available on the system it is running on.
Otherwise the program will crash.
This building technique is known as {\em weak linking} and allows great flexibility in code
development for various Mac OS X versions.

On the other hand, specifying a Deployment Target that is larger than the current SDK used 
is not allowed, as this would mean that the code could not run on the system which it was 
built for.

The deployment target is set using the {\tt MACOSX\_DEPLOYMENT\_TARGET}
environment variable.
This environment variable also related to the {\tt gcc} option {\tt -mmacosx-version-min}.


%%%%%%%%%%%%%%%%%%%%%%%%%%%%%%%%%%%%%%%%%%%%%%%%%%%
% GammaLib configuration
%%%%%%%%%%%%%%%%%%%%%%%%%%%%%%%%%%%%%%%%%%%%%%%%%%%
\section{\this\ configuration}

\this\ supports building of universal libraries in order to cope with the various Mac OS X versions
that actually do exist.
The {\tt configure} script will examine that actual system and will optionally define the
SDK version, the build architectures and the Deployment Target version.
For this purpose, the {\tt configure} options {\tt --enable-universalsdk} and {\tt --with-universal-archs}
have been added that allow tuning the actual settings.
The respective code in {\tt configure.ac} has been largely inspired from the Python {\tt ./configure} 
script, although it has been adjusted to fit the needs of \this.

%%%%%%%%%%%%%%%%%%%%%%%%%%%%%%%%%%%%%%%%%%%%%%%%%%%
\subsection{Native build}

First we consider the case of a native build, which means that code is only generated for a single
architecture, i.e. no universal library is created.
A native build is invoked when the configure option {\tt --enable-universalsdk} is {\bf not} given.

In this case, no SDK will be used, and the code will be built and linked against the libraries that
are available on the actual system.
In particular, no {\tt -isysroot} option will be given to {\tt gcc}.

If the library is built on Mac OS X $\le10.2$, the
{\tt MACOSX\_DEPLOYMENT\_TARGET}
environment variable for building and linking the \this\ shared library will be set to the actual 
Mac OS X version.
For Mac OS X $>10.2$, {\tt MACOSX\_DEPLOYMENT\_TARGET} will depend on the actual
architecture of the development platform.
For a PowerPC processor, {\tt MACOSX\_DEPLOYMENT\_TARGET=10.3}, while for an Intel
processor, {\tt MACOSX\_DEPLOYMENT\_TARGET=10.4} (Mac OS X is the first system that
supports Intel processors).


%%%%%%%%%%%%%%%%%%%%%%%%%%%%%%%%%%%%%%%%%%%%%%%%%%%
\subsection{Universal build}

Second, we consider the case of a universal build.
A universal build is invoked using
\begin{verbatim}
./configure --enable-universalsdk
\end{verbatim}

Universal builds were introduced with Mac OS X 10.4, yet it is not clear to me whether they also
would work on earlier systems, provided that PowerPC support has been added.
I have found universal binaries (built for {\tt i386} and {\tt ppc}) that were claimed
to work at least for Mac OS X 10.3, hence it is well possible that they work even for earlier versions.
Also Xcode 3 (the version on Mac OS X 10.6) seems to allow building universal binaries for 
Deployment Targets as low as 10.1.

Intel {\tt i386} has been introduced in Mac OS X 10.4, hence any Intel only library implies a
Deployment Target of at least 10.4.

Intel {\tt x86\_64} has been officially added in Mac OS X 10.5, but Xcode 3 allows
also the generation of {\tt x86\_64} code using the 10.4 SDK.
I have encountered problems, however, when compiling {\tt x86\_64} code against the 
10.4 SDK, such as a missing {\tt x86\_64} architecture in the system's {\tt readline}
library.
Thus, building for the {\tt x86\_64} architecture seems to require at least a Deployment Target 
of 10.5.

The \this\ {\tt configure} script follows here the approach taken in the Python {\tt configure} script,
although the detailed parametrisation is different.
For 32-bit builds, Mac OS X 10.4 is the default SDK, while 64-bit builds require Mac OS X 10.5.
These defaults are taken if {\tt --enable-universalsdk} is supplied without any argument.
A specific SDK directory may be specified using
\begin{verbatim}
./configure --enable-universalsdk=/Developer/SDKs/MacOSX10.6.sdk
\end{verbatim}
Furthermore,
\begin{verbatim}
./configure --enable-universalsdk=/
\end{verbatim}
will take the native systems headers and libraries as SDK.

{\tt configure} will use this information to append {\tt -isysroot \$\{UNIVERSALSDK\}}
options to {\tt CPPFLAGS}, {\tt LDFLAGS} and {\tt CXXFLAGS},
where {\tt \$\{UNIVERSALSDK\}} is the path to the specified SDK.
In addition, if the Mac OS X 10.4 SDK has been selected, {\tt gcc} version 4.0 will be used 
for compilation.
This is accomplished by setting
{\tt CC=gcc-4.0},
{\tt CXX=g++-4.0}, and
{\tt CPP=cpp-4.0}.

The target architectures that will be compiled in the universal libraries can be specified using\break
{\tt --with-universal-archs=VALUE}.
Table \ref{table:options} lists the {\tt VALUE}s that are available.
If  the\break {\tt --with-universal-archs} option is not given, {\tt 3-way} ({\tt ppc}, {\tt i386}, {\tt x86\_64})
will be assumed as default option.
The {\tt configure} script makes use of this information to add {\tt -arch} compile flags
to {\tt CXXFLAGS} and {\tt LDFLAGS} to define the architectures for which the code will be 
compiled.

%%% architecture options %%%%%%%%%%%%%%%%%%%%%%%%%%%%%%%%%%%%
\begin{table}[!h]
  \center
  \begin{tabular}{llc}
  \hline
  VALUE & Architectures & {\tt MACOSX\_DEPLOYMENT\_TARGET} \\
  \hline
  {\tt 32-bit} & {\tt ppc}, {\tt i386} & 10.4 \\
  {\tt 64-bit} & {\tt ppc64}, {\tt x86\_64} & 10.5 \\
  {\tt 3-way} (default) & {\tt ppc}, {\tt i386}, {\tt x86\_64} & 10.5 \\
  {\tt intel} & {\tt i386} and {\tt x86\_64} & 10.5 \\
  {\tt all} & {\tt ppc}, {\tt ppc64}, {\tt i386}, {\tt x86\_64} & 10.5 \\
  \hline
  \end{tabular}
  \caption{Possible values for the {\tt --with-universal-archs=VALUE} option.}
  \label{table:options}
\end{table}
%%%%%%%%%%%%%%%%%%%%%%%%%%%%%%%%%%%%%%%%%%%%%%%%%%%

The Deployment Target also depends on the specific combination of architectures.
{\tt i386} became only available in Mac OS X 10.4, hence {\tt 32-bit} builds require
a Deployment Target of 10.4.
{\tt x86\_64} became available in Mac OS X 10.5, hence all combinations containing
the {\tt x86\_64} architecture need a Deployment Target of 10.5.
The values taken for {\tt MACOSX\_DEPLOYMENT\_TARGET} are summarized in the 
last column of Table \ref{table:options}.
Note that the {\tt 32-bit} setting differs from what is found in Python's {\tt configure} script
(which assumes a Deployment Target of 10.3 for this build).

Note further that these values only apply if the build system is Mac OS X $\ge10.3$.
For earlier versions, the {\tt configure} script will set the Deployment Target to the Mac OS X
version.
I'm not sure that this is correct, but that's how the Python {\tt configure} script deals with earlier
Mac OS X versions, hence I just copied this part.


%%%%%%%%%%%%%%%%%%%%%%%%%%%%%%%%%%%%%%%%%%%%%%%%%%%
\subsection{Dependency libraries}
\label{sec:dependencies}

Although \this\ is designed to have as few dependencies as possible, it requires at least the {\tt cfitsio}
library to be installed to allow for FITS file access.
Furthermore, \this\ can make use of the {\tt readline} library (and the related and required {\tt ncurses}
library), but not having those will not restrict the functionalities of \this.

Unless we make an install from source of these dependency libraries, we have little control over the
actual architectures that they support.
In the best case, the libraries are provided as universal builds, and \this\ can link with the relevant
architectures.
{\tt readline} and {\tt ncurses} are in fact shipped with Mac OS X, and as we have shown by using the
{\tt file} command above, various architectures are supported (cf.~Table \ref{table:readline}).

%%% readline architectures %%%%%%%%%%%%%%%%%%%%%%%%%%%%%%%%%%%%
\begin{table}[!h]
  \center
  \begin{tabular}{ll}
  \hline
  SDK & Architectures \\
  \hline
  10.4 & {\tt ppc}, {\tt i386} \\
  10.5 & {\tt ppc}, {\tt i386}, {\tt ppc64}, {\tt x86\_64} \\
  10.6 & {\tt ppc}, {\tt i386}, {\tt x86\_64} \\
  \hline
  \end{tabular}
  \caption{Available architectures for {\tt readline} and {\tt ncurses} in different SDKs.}
  \label{table:readline}
\end{table}
%%%%%%%%%%%%%%%%%%%%%%%%%%%%%%%%%%%%%%%%%%%%%%%%%%%

The  bigger problem is {\tt cfitsio}, which is not shipped with Mac OS X, and which therefore can
exist in various architectures on a given system.
If the library is installed via MacPorts, it will only exist in the native architecture.
If it is installed from source, it is built by default as a universal library for {\tt i386} and {\tt x86\_64}.
{\bf The fact that there is no a priori control over the architectures that are available in {\tt cfitsio} is
a major flaw of the \this\ installation on Mac OS X.}


%%%%%%%%%%%%%%%%%%%%%%%%%%%%%%%%%%%%%%%%%%%%%%%%%%%
\subsection{Python module}
\label{sec:python}

A even worse flaw exists for building the \this\ Python module, because a common architecture
match has to be found between the Python executable, the \this\ library, and the dependence
libraries ({\tt cfitsio}, {\tt readline}, and {\tt ncurses}).

Python modules need in principle to be built for all architectures that are present in the Python
executable.
This is automatically guaranteed by using the {\tt distutils} Python module for compilation,
linking and installation, which will extract the relevant information from the actual Python
build.
\this\ makes use of this module, which means that the SDK and the Deployment Target 
environment variable
{\tt MACOSX\_DEPLOYMENT\_TARGET}
that were used for compiling the \this\ library will {\bf not} be used for the compilation of the \this\ 
Python module.
The configuration of the \this\ Python module will be fully determined by the configuration
of the Python executable.

The \this\ Python module, however, relies on the {\tt libgamma.dylib} library, which in turn depends
on the {\tt libcfitsio.dylib} (and eventually also the {\tt libreadline.dylib} and {\tt libncurses.dylib})
libraries.
The \this\ Python module will thus only work if the relevant architecture is indeed found in all
these libraries.
This does not mean that all possible architectures need to be present, but if Python is for
example launched as an Intel 64-bit application it needs to find {\tt x86\_64} code in
{\tt libgamma.dylib}, {\tt libcfitsio.dylib}, {\tt libreadline.dylib} and {\tt libncurses.dylib}.


%%%%%%%%%%%%%%%%%%%%%%%%%%%%%%%%%%%%%%%%%%%%%%%%%%%
\subsection{Other issues}

Another important issue arose related to dependencies that are automatically created by the
{\tt automake} system.
These dependencies get created by default when
{\tt AM\_INIT\_AUTOMAKE}
is specified in {\tt configure.ac}.
Dependency creation adds a number of compile flags {\tt -M} that are incompatible with building
universal libraries.
To avoid this problem,
{\tt AM\_INIT\_AUTOMAKE([no-dependencies])}
has been specified in {\tt configure.ac}.


%%%%%%%%%%%%%%%%%%%%%%%%%%%%%%%%%%%%%%%%%%%%%%%%%%%
% Case studies
%%%%%%%%%%%%%%%%%%%%%%%%%%%%%%%%%%%%%%%%%%%%%%%%%%%
\section{Case studies}

In order to test the \this\ package on Mac OS X and to explore possible limitations of the build
system, a number of case studies have been done.
All tests have been performed using a development version of {\tt GammaLib-00-05-00}.
The test platform was a Intel Core i7 MacBook Pro, running Mac OS X 10.6.8
which by default compiles code for the {\tt x86\_64} architecture. 
Apple's SDKs 10.4, 10.5 and 10.6 have been installed.


%%%%%%%%%%%%%%%%%%%%%%%%%%%%%%%%%%%%%%%%%%%%%%%%%%%
\subsection{Case A: Installing \this\ on vanilla Mac OS X}

With vanilla Mac OS X, I mean an operating system that comes out of the box, without any
3rd party adds, such as different Python versions or specific {\tt cfitsio}, {\tt readline}, or
{\tt ncurses} libraries.

The Mac OS X 10.6 used for testing came with Python 2.5 and Python 2.6, and tests have been executed
for both versions.
The major difference between both versions is that Python 2.5 is a {\tt 32-bit} build while
Python 2.6 is a {\tt 3-way} build (cf.~Table \ref{table:macpython}).

%%% Python architectures %%%%%%%%%%%%%%%%%%%%%%%%%%%%%%%%%%%%%%
\begin{table}[!h]
  \center
  \begin{tabular}{ll}
  \hline
  Python & Architectures \\
  \hline
  2.5 & {\tt ppc}, {\tt i386} \\
  2.6 & {\tt ppc}, {\tt i386}, {\tt x86\_64} \\
  \hline
  \end{tabular}
  \caption{Architectures supported by Mac OS X provided Python (for Mac OS X 10.6).}
  \label{table:macpython}
\end{table}
%%%%%%%%%%%%%%%%%%%%%%%%%%%%%%%%%%%%%%%%%%%%%%%%%%%

Prior to installing \this, {\tt cfitsio} needed to be installed from source.
{\tt cfitsio} version 3.280 was installed in {\tt /usr/local/gamma} using 
{\small\begin{verbatim}
$ ./configure --prefix=/usr/local/gamma
$ make shared
$ make install
\end{verbatim}}
This installed {\tt cfitsio} as universal library built for the {\tt i386} and {\tt x86\_64} architectures,
which is the default configuration.


%%%%%%%%%%%%%%%%%%%%%%%%%%%%%%%%%%%%%%%%%%%%%%%%%%%
\subsubsection{Building against Python 2.5}

Python 2.5 was first tested using the following command sequence:
{\small\begin{verbatim}
$ export PATH=/System/Library/Frameworks/Python.framework/Versions/2.5/bin:/usr/bin:/bin:/usr/sbin:/sbin
$ ./configure
$ make
$ make check
\end{verbatim}}

During the build step (i.e. after typing {\tt make}), the following warnings appeared at the stage
where the shared libraries of the Python modules were linked against {\tt libgamma.dylib}:
{\small\begin{verbatim}
ld: warning: in ../src/.libs/libgamma.dylib, file is not of required architecture
ld: warning: in ../src/.libs/libgamma.dylib, file is not of required architecture
ld: warning: in /usr/local/gamma/lib/libcfitsio.dylib, missing required architecture ppc in file
\end{verbatim}}
The reason for these warnings is obvious.
{\tt x86\_64} is the native architecture of the test platform, and the \this\ shared library
{\tt libgamma.dylib} has been compiled for this architecture.
Python 2.5, however, has been compiled for {\tt ppc} and {\tt i386}, and consequently,
the \this\ Python modules will be built for the same architectures.
Linking the Python modules against {\tt libgamma.dylib} then fails, as the library neither
contains the {\tt ppc} nor the {\tt i386} architecture.
Furthermore, linking against {\tt libcfitsio.dylib} for the {\tt ppc} architecture also fails,
as {\tt cfitsio} has been compiled for Intel architectures only, and thus does not provide
code for {\tt ppc}.

{\tt make check} works then fine for the tests that verify the {\tt libgamma.dylib} itself (which
are the first 14 tests), but fails on the final verification of the Python module.
The following Python run-time error occurs:
{\small\begin{verbatim}
Traceback (most recent call last):
  File "./test_python.py", line 3, in <module>
    from gammalib import *
  File "/Users/jurgen/dev/gammalib/pyext/gammalib/__init__.py", line 1, in <module>
    from app import *
  File "/Users/jurgen/dev/gammalib/pyext/gammalib/app.py", line 7, in <module>
    import _app
ImportError: dlopen(/Users/jurgen/dev/gammalib/pyext/gammalib/_app.so, 2): 
Symbol not found: __ZNK17GExceptionHandler4whatEv
  Referenced from: /Users/jurgen/dev/gammalib/pyext/gammalib/_app.so
  Expected in: flat namespace
 in /Users/jurgen/dev/gammalib/pyext/gammalib/_app.so
FAIL: test_python.py
===================================
1 of 15 tests failed
Please report to knodlseder@cesr.fr
===================================
\end{verbatim}}
This error is related to the attempt of the Python module {\tt \_app.so} to load
a symbol from {\tt libgamma.dylib}, which obviously fails because the symbol is
not found for the requested architecture.

The problem can be solved by building \this\ as universal library.
The following sequence
{\small\begin{verbatim}
$ ./configure --enable-universalsdk
$ make
$ make check
\end{verbatim}}
builds \this\ as {\tt 32-bit} universal binary, compiling for the {\tt ppc} and {\tt i386}
architectures.
These are the same architectures that are also present in the Python executable.

Building leaves
{\small\begin{verbatim}
ld: warning: in /usr/local/gamma/lib/libcfitsio.dylib, missing required architecture ppc in file
\end{verbatim}}
as the only warning during compile time, because there is still no {\tt ppc} architecture in
{\tt cfitsio}.
{\tt make check} was now successful for all 15 unit tests, including the Python test.


%%%%%%%%%%%%%%%%%%%%%%%%%%%%%%%%%%%%%%%%%%%%%%%%%%%
\subsubsection{Building against Python 2.6}

Python 2.6 was tested using the following command sequence:
{\small\begin{verbatim}
$ export PATH=/System/Library/Frameworks/Python.framework/Versions/2.6/bin:/usr/bin:/bin:/usr/sbin:/sbin
$ ./configure
$ make
$ make check
\end{verbatim}}
which compiled \this\ for the native {\tt x86\_64} architecture.
During the build of the Python modules the following warnings occurred
{\small\begin{verbatim}
ld: warning: in ../src/.libs/libgamma.dylib, file is not of required architecture
ld: warning: in ../src/.libs/libgamma.dylib, file is not of required architecture
ld: warning: in /usr/local/gamma/lib/libcfitsio.dylib, missing required architecture ppc in file
\end{verbatim}}
which are related to the fact that {\tt libgamma.dylib} does not contain the
{\tt i386} and {\tt ppc} architectures, and {\tt libcfitsio.dylib} does not contain the
{\tt ppc} architecture.
Despite these warnings, all 15 unit tests were executed successfully.

I also tested Python 2.6 using
{\small\begin{verbatim}
$ ./configure --enable-universalsdk
$ make
$ make check
\end{verbatim}}
The only warning occurring now is
{\small\begin{verbatim}
ld: warning: in /usr/local/gamma/lib/libcfitsio.dylib, missing required architecture ppc in file
\end{verbatim}}
and all 15 unit tests were executed successfully.


%%%%%%%%%%%%%%%%%%%%%%%%%%%%%%%%%%%%%%%%%%%%%%%%%%%
\subsubsection{Summary}

The test results for the vanilla Mac OS X are summarised in Table \ref{table:vanilla}.
The column {\tt libgamma} specifies the architectures that are compiled into the \this\
library, while the column {\tt gammalib} gives the architectures available in the
Python module.
The columns ``use as API'' and ``use from Python'' specify the architectures that are effectively
used when \this\ is used as a C++ API and when it is used from Pyhton as a module, respectively.
For both Python version, the {\tt configure} option {\tt --enable-universalsdk} allows a successful
build of \this.

%%% Python architectures %%%%%%%%%%%%%%%%%%%%%%%%%%%%%%%%%%%%%%
\begin{table}[!h]
  \small
  \center
  \begin{tabular}{cc|cccc}
  \hline
   & & \multicolumn{4}{c}{Architectures} \\
  Python & {\tt configure} flags & {\tt libgamma} & {\tt gammalib} & use as API & use from Python \\
  \hline
  2.5 & - & {\tt x86\_64} & {\tt ppc}, {\tt i386} & {\tt x86\_64} & {\bf failed} \\
  2.5 & --enable-universalsdk & {\tt ppc}, {\tt i386}, {\tt x86\_64} & {\tt ppc}, {\tt i386} & {\tt x86\_64} & {\tt i386} \\
  2.6 & - & {\tt x86\_64} & {\tt ppc}, {\tt i386}, {\tt x86\_64} & {\tt x86\_64} & {\tt x86\_64} \\
  2.6 & --enable-universalsdk & {\tt ppc}, {\tt i386}, {\tt x86\_64} & {\tt ppc}, {\tt i386}, {\tt x86\_64} & {\tt x86\_64} & {\tt x86\_64} \\
  \hline
  \end{tabular}
  \caption{Summary of vanilla Mac OS X tests.}
  \label{table:vanilla}
\end{table}
%%%%%%%%%%%%%%%%%%%%%%%%%%%%%%%%%%%%%%%%%%%%%%%%%%%


%%%%%%%%%%%%%%%%%%%%%%%%%%%%%%%%%%%%%%%%%%%%%%%%%%%
\subsection{Case B: Pure MacPorts system}

A popular source for software on Mac OS X is MacPorts ({\tt http://www.macports.org/}).
Software from MacPorts is compiled upon installation from source, and by default this is only
done for the architecture the system is actually running on ({\tt x86\_64} is our case).\footnote{
There exists the possibility of installing universal builds via MacPorts, but I assume here that 
most people will not use this option.}

The software installed from MacPorts was Python (versions 2.5, 2.6 and 2.7), {\tt cfitsio},
{\tt readline} and {\tt ncurses}.
I tested MacPorts using
{\small\begin{verbatim}
$ export PATH=/opt/local/Library/Frameworks/Python.framework/Versions/2.X/bin:/usr/bin:/bin:/usr/sbin:/sbin
$ ./configure
$ make
$ make check
\end{verbatim}}
(where {\tt X} stands for the minor Python version number).
No warning has been issued during the build, and all 15 tests passed successfully.

It should be noted that the \this\ library was actually linked against the {\tt readline} and {\tt ncurses}
libraries provided by Mac OS X, as those are found earlier in the list of searched paths (MacPorts
{\tt /opt/local/lib} directory is only searched last).
The Python module linker, however, had {\tt -L/opt/local/lib} as first search path in the list of
options (probably because the Python library is installed under this directory), and consequently
the \this\ Python modules were linked against MacPorts {\tt readline} and {\tt ncurses} libraries.


%%%%%%%%%%%%%%%%%%%%%%%%%%%%%%%%%%%%%%%%%%%%%%%%%%%
\subsection{Case C: MacPorts {\tt cfitsio} and Mac OS X Python}


%%%%%%%%%%%%%%%%%%%%%%%%%%%%%%%%%%%%%%%%%%%%%%%%%%%
\subsection{Case D: MacPorts {\tt cfitsio} and Python from {\tt http://www.python.org/}}

{\tt http://www.python.org/} provides Python binary installers for Mac OS X that come generally
in two versions:
\begin{itemize}
\item 64-bit/32-bit {\tt x86\_64}/{\tt i386} installer (corresponding to our option {\tt intel})
\item 32-bit {\tt ppc}/{\tt i386} installer (corresponding to our option {\tt 32-bit})
\end{itemize}



%%%%%%%%%%%%%%%%%%%%%%%%%%%%%%%%%%%%%%%%%%%%%%%%%%%
\subsection{Software}

The test matrix has been applied to a development version of {\tt GammaLib-00-05-00}.

\this\ depends primarily on cfitsio, and at least one of the architectures that is compiled into \this\
has to match the one cfitsio was built for.
If cfitsio is installed via MacPorts, cfitsio is built for the native architecture.
On the computer on which the tests were performed, this was {\tt x86\_64}.

\this\ provides bindings to Python, but they will only work if at least one of the architectures that 
are compiled into \this\ matches one of the architectures in Python, and if this match is also found
in cfitsio (and eventually readline and ncurses).
{\bf There is an obvious example for which Python support will never fully work, which is if none 
of the architectures in Python matches the one in cfitsio} (in this case any FITS related function
will result in a runtime error).
There is nothing that can be done about this at the level of \this.
The only solution is the installation of cfitsio as a universal library, comprising the required
architectures.

In this series of tests we explored the worst case scenario, making use of the cfitsio MacPorts
install which is only compiled for {\tt x86\_64}.
The cfitsio version was 3.26.


%%%%%%%%%%%%%%%%%%%%%%%%%%%%%%%%%%%%%%%%%%%%%%%%%%%
\subsection{Procedure}

Tests consisted of building and checking \this\ using different Python versions and \this\
configuration options.
Typically, the test sequence was comprised by the following commands:
\begin{verbatim}
$ export PATH=/Users/jurgen/Software/python/python27/<SDK>/<arch>/bin:/usr/bin:/bin:
/usr/sbin:/sbin:/opt/local/bin
$ make distclean
$ ./configure <optionn>
$ make
$ make check
\end{verbatim}

The {\tt make distclean} step guarantees that results from previous tests will be erased before
starting a new test.

For the purpose of these tests, a number of different versions of Python 2.7.2 have been compiled,
using different SDKs, DTs and architectures.
They were installed in different directories, so that specifying a specific directory before the
{\tt ./configure} step will force \this\ to use this specific Python version.
Note that an Intel build requires a Mac OS X Deployment Target of $\ge10.4$, which obviously
is not satisfied for a number of Python configurations (they are marked by $\dag$ in 
Table \ref{table:tests}).
This is an obvious bug in the Python 2.7.2 {\tt ./configure} script, leading to numerous
warnings
\begin{verbatim}
#warning Building for Intel with Mac OS X Deployment Target < 10.4 is invalid.
\end{verbatim}
during compilation.
These versions have been included in the test matrix as they may exist on some systems.

Furthemore, not all configuration options worked.
Several options led to
\begin{verbatim}
checking size of size_t... configure: error: in `/Users/jurgen/Software/python/src/Python-2.7.2':
configure: error: cannot compute sizeof (size_t)
\end{verbatim}
during the {\tt ./configure} step of Python.
In particular, the {\tt ppc64} target did only configure for SDK 10.5 with success.
Furthemore, SDK 10.4 also could not be configured to include the {\tt x86\_64} target.

In addition to Python 2.7.2 a number of other Python versions were also tested.


%%%%%%%%%%%%%%%%%%%%%%%%%%%%%%%%%%%%%%%%%%%%%%%%%%%
\subsection{Case B: MacPorts {\tt cfitsio} and Mac OS X Python}

I have no clue what in general is shipped, and how things evolve from one Mac OS X version to another.

%%% Python architectures %%%%%%%%%%%%%%%%%%%%%%%%%%%%%%%%%%%%%%
\begin{table}[!h]
  \center
  \begin{tabular}{ll}
  \hline
  Python & Architectures \\
  \hline
  2.5 & {\tt ppc}, {\tt i386} \\
  2.6 & {\tt ppc}, {\tt i386}, {\tt x86\_64} \\
  \hline
  \end{tabular}
  \caption{Architectures supported by Mac OS X provided Python (for Mac OS X 10.6).}
  \label{table:macpython}
\end{table}
%%%%%%%%%%%%%%%%%%%%%%%%%%%%%%%%%%%%%%%%%%%%%%%%%%%






%%%%%%%%%%%%%%%%%%%%%%%%%%%%%%%%%%%%%%%%%%%%%%%%%%%
\subsection{Results}

The test results are summarized in Table \ref{table:tests}.

%%% Test matrix %%%%%%%%%%%%%%%%%%%%%%%%%%%%%%%%%%%%%%%%%%
\begin{table}[!t]
  \center
  \begin{tabular}{cccc|cc|ccc|c}
  \hline
  \multicolumn{4}{c|}{Python} & \multicolumn{6}{c}{\this} \\
  Version & SDK & DT & Architecture(s) & sdk$^\ast$ & archs$^{\ast\ast}$ & SDK & DT & Architecture(s) & result \\
  \hline
  2.7 & 10.4 & 10.3$^\dag$ & {\tt ppc}, {\tt i386} & - & - & / & 10.4 & {\tt x86\_64} & Failure 1 \\
  2.7 & 10.4 & 10.3$^\dag$ & {\tt ppc}, {\tt i386} & yes & - & 10.4 & 10.4 & {\tt ppc}, {\tt i386} & Failure 2 \\
  \hline
  \hline
  \hline
  2.5 & / & 10.6 & {\tt ppc}, {\tt i386} & - & - & / & 10.4 & {\tt x86\_64} & fails$^{(1)}$ \\
  2.5 & / & 10.6 & {\tt ppc}, {\tt i386} & yes & - & 10.4 & 10.4 & {\tt ppc}, {\tt i386} & ok \\
  \hline
  2.6 & / & 10.6 & {\tt x86\_64} & - & - & / & 10.4 & {\tt x86\_64} & ok \\
  2.6 & / & 10.6 & {\tt x86\_64} & / & {\tt 3-way} & / & 10.5 & {\tt ppc}, {\tt i386}, {\tt x86\_64} & ok$^{(3)}$ \\
  2.6 & / & 10.6 & {\tt ppc}, {\tt i386}, {\tt x86\_64} & / & - & / & 10.4 & {\tt x86\_64} & ok \\
  \hline
  2.7 & / & 10.4 & {\tt x86\_64} & - & - & / & 10.4 & {\tt x86\_64} & ok?  \\
  \hline
  2.7 & 10.4 & 10.3$^\dag$ & {\tt ppc}, {\tt i386} & - & - & / & 10.4 & {\tt x86\_64} & fails$^{(1)}$  \\
  2.7 & 10.4 & 10.3$^\dag$ & {\tt ppc}, {\tt i386} & yes & - & 10.4 & 10.4 & {\tt ppc}, {\tt i386} & ok? \\
  2.7 & 10.4 & 10.3$^\dag$ & {\tt ppc}, {\tt i386} & / & - & / & 10.4 & {\tt ppc}, {\tt i386} &  ok \\
  2.7 & 10.4 & 10.3$^\dag$ & {\tt ppc}, {\tt i386} & 10.5 & - & 10.5 & 10.4 & {\tt ppc}, {\tt i386} & ok \\
  2.7 & 10.4 & 10.3$^\dag$ & {\tt ppc}, {\tt i386} & 10.6 & - & 10.6 & 10.4 & {\tt ppc}, {\tt i386} & ok \\
  \hline
  2.7 & 10.5 & 10.3$^\dag$ & {\tt ppc}, {\tt i386} & - & - & / & 10.4 & {\tt x86\_64} & fails$^{(1)}$ \\
  2.7 & 10.5 & 10.3$^\dag$ & {\tt ppc}, {\tt i386} & yes & - & 10.4 & 10.4 & {\tt ppc}, {\tt i386} & ok \\
  2.7 & 10.5 & 10.5 & {\tt ppc64}, {\tt x86\_64} & - & - & / & 10.4 & {\tt x86\_64} & ok \\
  \hline
  2.7 & 10.6 & 10.3$^\dag$ & {\tt ppc}, {\tt i386} & - & - & / & 10.4 & {\tt x86\_64} & fails$^{(1)}$? \\
  2.7 & 10.6 & 10.3$^\dag$ & {\tt ppc}, {\tt i386} & yes & - & 10.4 & 10.4 & {\tt ppc}, {\tt i386} & ok$^{(2)}$ \\
  2.7 & 10.6 & 10.3$^\dag$ & {\tt ppc}, {\tt i386} & / & - & / & 10.4 & {\tt ppc}, {\tt i386} & ok$^{(2)}$ \\
  2.7 & 10.6 & 10.3$^\dag$ & {\tt ppc}, {\tt i386} & 10.5 & - & 10.5 & 10.4 & {\tt ppc}, {\tt i386} & ok$^{(2)}$ \\
  2.7 & 10.6 & 10.3$^\dag$ & {\tt ppc}, {\tt i386} & 10.6 & - & 10.6 & 10.4 & {\tt ppc}, {\tt i386} & ok$^{(2)}$ \\
  \hline
  2.7 & 10.6 & 10.5 & {\tt ppc}, {\tt i386}, {\tt x86\_64} & - & - & / & 10.4 & {\tt x86\_64} & ok \\
  \hline
  \end{tabular}
  \begin{flushleft}
  $^\ast${\tt --enable-universalsdk} option: not given (``-"), given (``yes"), use root directory as SDK 
  (``/"), use specific SDK (``10.X"). \\
  $^{\ast\ast}${\tt --with-universal-archs} option: not given (``-"), otherwise gives the specified value. \\
  $^{(1)}$Failure of python test in {\tt make check} due to different architectures for 
  Python and {\tt libgamma.dylib}.\\
  $^{(2)}$Warning about building for Intel with Mac OS X Deployment Target $<10.4$. \\
  $^{(3)}$Python interface was only built for the architecture(s) that are also present in the
  Python build.
  \end{flushleft}
  \caption{\task\ test matrix. 
  The first 4 columns given the Python configuration 
  (version, SDK, {\tt MACOSX\_DEPLOYMENT\_TARGET} and architecture).
  The next 2 columns give the values of the {\tt --enable-universalsdk} and 
  {\tt --with-universal-archs} options given to {\tt configure} of \this.
  The following 3 columns give the \this\ configuration (as derived from the configuration
  options), and the final column indicates the test results.}
  \label{table:tests}
\end{table}
%%%%%%%%%%%%%%%%%%%%%%%%%%%%%%%%%%%%%%%%%%%%%%%%%%%

Most tests were successful.
The few failures that occured were all related to differences in architecture between Python and
the \this\ shared library {\tt libgamma.dylib}.
These failures are marked by "fail$^{(1)}$" in the results column.
Typical error messages that occured in the {\tt make check} step when running the Python
test were
\begin{verbatim}
ImportError: dlopen(/Users/jurgen/dev/gammalib/pyext/build/_gammalib.so, 2): 
  Symbol not found: __ZNK17GExceptionHandler4whatEv
  Referenced from: /Users/jurgen/dev/gammalib/pyext/build/_gammalib.so
  Expected in: dynamic lookup
\end{verbatim}
or
\begin{verbatim}
ImportError: dlopen(/Users/jurgen/dev/gammalib/pyext/build/_gammalib.so, 2): 
no suitable image found.  
Did find: /Users/jurgen/dev/gammalib/pyext/build/_gammalib.so: 
no matching architecture in universal wrapper
\end{verbatim}
These errors indicate that the required achitecture was not found in the shared library
{\tt \_gammalib.so} which is loaded when {\tt import gammalib} is executed in Python.
The error occured for cases where Python was only compiled for {\tt ppc} and {\tt i386}, while
\this\ was compiled for {\tt x86\_64} (note that the Python bindings were however compiled
correctly for the {\tt ppc} and {\tt i386} archirectures).
Adding the \this\ configuration option {\tt --enable-universalsdk} always solved the problem
({\bf to be confirmed}).

Warnings: uniquement sur SDK 10.6

Note also that the Python interface is only built for the architectures that are also present in
Python (distuils).


%%%%%%%%%%%%%%%%%%%%%%%%%%%%%%%%%%%%%%%%%%%%%%%%%%%
\subsubsection{Failure 1}

While the library build was running without any problem, the following warnings appeared
during the build of the Python bindings:
\begin{verbatim}
ld: warning: in ../src/.libs/libgamma.dylib, file is not of required architecture
ld: warning: in /opt/local/lib/libcfitsio.dylib, file is not of required architecture
ld: warning: in /opt/local/lib/libreadline.dylib, file is not of required architecture
ld: warning: in /opt/local/lib/libncurses.dylib, file is not of required architecture
\end{verbatim}
All libraries that are listed here were compiled for {\tt x86\_64}, while the Python bindings
were compiled for {\tt i386} and {\tt ppc}.
The linker warning informs about that mismatch, the Python bindings were however built
nevertheless.

{\tt make check} terminates with
\begin{verbatim}
Traceback (most recent call last):
  File "./test_python.py", line 3, in <module>
    from gammalib import *
  File "/Users/jurgen/dev/gammalib/pyext/gammalib/__init__.py", line 1, in <module>
    from app import *
  File "/Users/jurgen/dev/gammalib/pyext/gammalib/app.py", line 7, in <module>
    import _app
ImportError: dlopen(/Users/jurgen/dev/gammalib/pyext/gammalib/_app.so, 2): Symbol not found:
 __ZNK17GExceptionHandler4whatEv
  Referenced from: /Users/jurgen/dev/gammalib/pyext/gammalib/_app.so
  Expected in: dynamic lookup

FAIL: test_python.py
===================================
1 of 15 tests failed
Please report to knodlseder@cesr.fr
===================================
\end{verbatim}
The reason for this error is that the shared library {\tt \_app.so} -- which is the Python extension
module -- has been built for {\tt i386} and {\tt ppc}, while the shared \this\ library {\tt libgamma.dylib}
has been built for {\tt x86\_64}. 
{\tt \_app.so} therefore is not able to find the symbol in the required architecture, and the
{\tt dlopen} function (which loads dynamic library modules) fails.


%%%%%%%%%%%%%%%%%%%%%%%%%%%%%%%%%%%%%%%%%%%%%%%%%%%
\subsubsection{Failure 2}

In this test, \this\ was compiled for {\tt i386} and {\tt ppc}, and the same architectures are
available in Python.

During {\tt make} the following warnings occur when creating the {\tt libgamma.dylib}
library and when building the Python bindings.
\begin{verbatim}
ld: warning: in /opt/local/lib/libcfitsio.dylib, file is not of required architecture
ld: warning: in /opt/local/lib/libreadline.dylib, file is not of required architecture
ld: warning: in /opt/local/lib/libncurses.dylib, file is not of required architecture
\end{verbatim}
As \this\ is now compiled for {\tt i386} and {\tt ppc}, there is no mismatch between
{\tt libgamma.dylib} and Python.
However, none of the 3rd party libraries contains the required architectures, hence using
3rd party services fails.

This can be seen during {\tt make check}, which gives the following errors:
\begin{verbatim}
***********************
* GFits class testing *
***********************
dyld: lazy symbol binding failed: Symbol not found: _ffopen
  Referenced from: /Users/jurgen/dev/gammalib/src/.libs/libgamma.0.dylib
  Expected in: dynamic lookup

dyld: Symbol not found: _ffopen
  Referenced from: /Users/jurgen/dev/gammalib/src/.libs/libgamma.0.dylib
  Expected in: dynamic lookup
\end{verbatim}
and
\begin{verbatim}
Traceback (most recent call last):
  File "./test_python.py", line 3, in <module>
    from gammalib import *
  File "/Users/jurgen/dev/gammalib/pyext/gammalib/__init__.py", line 1, in <module>
    from app import *
  File "/Users/jurgen/dev/gammalib/pyext/gammalib/app.py", line 7, in <module>
    import _app
ImportError: dlopen(/Users/jurgen/dev/gammalib/pyext/gammalib/_app.so, 2): Symbol not found: _readline
  Referenced from: /Users/jurgen/dev/gammalib/pyext/gammalib/_app.so
  Expected in: dynamic lookup

FAIL: test_python.py
===================================
6 of 15 tests failed
Please report to knodlseder@cesr.fr
===================================
\end{verbatim}
The first test that failed was the {\tt GFits} test, as no FITS support was available.


%%%%%%%%%%%%%%%%%%%%%%%%%%%%%%%%%%%%%%%%%%%%%%%%%%%
% Conclusions
%%%%%%%%%%%%%%%%%%%%%%%%%%%%%%%%%%%%%%%%%%%%%%%%%%%
\section{Conclusions}

Deploying \this\ on Mac OS X can become a major challenge, as result of the various systems
and CPU architectures that may be involved.

If the user has full control over the components in the system, he can tweak the \this\ {\tt configure}
flags to setup the system so that it matches the Python and {\tt cfitsio} architectures.
In the worst case, the user may have to compile {\tt cfitsio} from source, using the same architectures
that are also present in Python, so that \this, Python and {\tt cfitsio} work seamlessly together.
In this case, the user should first determine the architectures present in Python using
\begin{verbatim}
$ file `which python`
\end{verbatim}


%%%%%%%%%%%%%%%%%%%%%%%%%%%%%%%%%%%%%%%%%%%%%%%%%%%
% Appendix
%%%%%%%%%%%%%%%%%%%%%%%%%%%%%%%%%%%%%%%%%%%%%%%%%%%
\section{Appendix}

%%%%%%%%%%%%%%%%%%%%%%%%%%%%%%%%%%%%%%%%%%%%%%%%%%%
\subsection{Building a universal {\tt cfitsio} library}

A universal {\tt cfitsio} library is built on Mac OS X by using the following sequence of commands
{\small
\begin{verbatim}
$ ./configure CFLAGS="-arch i386 -arch ppc -arch x86_64" LDFLAGS="-arch i386 -arch ppc -arch x86_64"
$ make shared
\end{verbatim}}
Note that by default {\tt cfitsio} adds {\tt -arch i386 -arch x86\_64} to the compile flags, hence the shorter
{\small
\begin{verbatim}
$ ./configure CFLAGS="-arch ppc" LDFLAGS="-arch ppc"
$ make shared
\end{verbatim}}
leads to the same result.
The resulting shared library will then have the following architectures:
{\small
\begin{verbatim}
$ file libcfitsio.dylib
libcfitsio.dylib: Mach-O universal binary with 3 architectures
libcfitsio.dylib (for architecture i386):	Mach-O dynamically linked shared library i386
libcfitsio.dylib (for architecture x86_64):	Mach-O 64-bit dynamically linked shared library x86_64
libcfitsio.dylib (for architecture ppc7400):	Mach-O dynamically linked shared library ppc
\end{verbatim}}


\end{document}

