%%%%%%%%%%%%%%%%%%%%%%%%%%%%%%%%%%%%%%%%%%%%%%%%%%%
% Handling of Regions of Interest in stacked analysis
%%%%%%%%%%%%%%%%%%%%%%%%%%%%%%%%%%%%%%%%%%%%%%%%%%%

%%%%%%%%%%%%%%%%%%%%%%%%%%%%%%%%%%%%%%%%%%%%%%%%%%%
% Definitions for manual package
%%%%%%%%%%%%%%%%%%%%%%%%%%%%%%%%%%%%%%%%%%%%%%%%%%%
\newcommand{\task}{\mbox{GammaLib}}
\newcommand{\this}{\mbox{\tt \task}}
\newcommand{\shorttype}{\mbox{}}
\newcommand{\doctype}{\mbox{RoI handling in stacked analysis}}
\newcommand{\reference}{\mbox{GammaLib-TN0002}}
\newcommand{\version}{\mbox{1.0}}
\newcommand{\calendar}{\mbox{20 July 2016}}
\newcommand{\auth}{\mbox{J\"urgen Kn\"odlseder}}
\newcommand{\approv}{\mbox{J\"urgen Kn\"odlseder}}


%%%%%%%%%%%%%%%%%%%%%%%%%%%%%%%%%%%%%%%%%%%%%%%%%%%
% Document definition
%%%%%%%%%%%%%%%%%%%%%%%%%%%%%%%%%%%%%%%%%%%%%%%%%%%
\documentclass{article}[12pt,a4]
\usepackage{epsfig}
\usepackage{technote}


%%%%%%%%%%%%%%%%%%%%%%%%%%%%%%%%%%%%%%%%%%%%%%%%%%%
% Begin of document body
%%%%%%%%%%%%%%%%%%%%%%%%%%%%%%%%%%%%%%%%%%%%%%%%%%%
\begin{document}
\frontpage


%%%%%%%%%%%%%%%%%%%%%%%%%%%%%%%%%%%%%%%%%%%%%%%%%%%
% Scope
%%%%%%%%%%%%%%%%%%%%%%%%%%%%%%%%%%%%%%%%%%%%%%%%%%%
\section{Scope}

This technical note provides considerations for handling of Regions of Interest in a stacked
analysis.


%%%%%%%%%%%%%%%%%%%%%%%%%%%%%%%%%%%%%%%%%%%%%%%%%%%
% Formulating the problem
%%%%%%%%%%%%%%%%%%%%%%%%%%%%%%%%%%%%%%%%%%%%%%%%%%%
\section{Formulating the problem}

Each analysis of an observation requires the definition of a Region of Interest (RoI) which delimits
the data space that is used for the analysis.
The RoI can be a spatial selection (for example a circle on the sky) or an energy selection, or a
combination of both.
Let's denote in the following the data space coordinates by $\vec{d}$.
The expected number of events in an observation $i$ after RoI selection is then given by
\begin{equation}
e_i(\vec{d}) = w_i(\vec{d}) \times M_i(\vec{d})
\end{equation}
where $M_i(\vec{d})$ is the expected number of events (computed from a model) before the
RoI selection, and $w_i(\vec{d})$ is the RoI window function that is $1$ inside the RoI and $0$ 
outside.
For stacked analysis the data space is represented by a counts cube, and the RoI can only partially
overlap with some of the cube bins.
The partial overlap can for example be spatially or in energy.
The window function can then be generalised into an function given the fractional overlap of a
counts cube bin with the RoI, and this function takes values between $0$ and $1$.

Let's now consider the stacking of two observations into a single counts cube (the reasoning can
be easily extended to an arbitrary number of observations).
The expected number of events in the stacked counts cube is then given by
\begin{equation}
e(\vec{d}) = w_1(\vec{d}) M_1(\vec{d}) + w_2(\vec{d}) M_2(\vec{d}) \, .
\label{eq:added}
\end{equation}
We are now seeking for a weighting function $w(\vec{d})$ of the stacked observation so that
we can simply write the expected number of events as the product between the weighting
function and a stacked model, i.e.
\begin{equation}
e(\vec{d}) = w(\vec{d}) \times \left( M_1(\vec{d}) + M_2(\vec{d})\right) \, .
\label{eq:stacked}
\end{equation}
Combining Eq.~(\ref{eq:added}) and Eq.~(\ref{eq:stacked}) results in
\begin{eqnarray}
w(\vec{d}) \times \left( M_1(\vec{d}) + M_2(\vec{d})\right) & = & w_1(\vec{d}) M_1(\vec{d}) + w_2(\vec{d}) M_2(\vec{d}) \nonumber \\
w(\vec{d}) & = & \frac{w_1(\vec{d}) M_1(\vec{d}) + w_2(\vec{d}) M_2(\vec{d})} {M_1(\vec{d}) + M_2(\vec{d})} \nonumber \\
w(\vec{d}) & = & w_1(\vec{d}) \frac{M_1(\vec{d})}{M_1(\vec{d}) + M_2(\vec{d})} +
w_2(\vec{d}) \frac{M_2(\vec{d})}{M_1(\vec{d}) + M_2(\vec{d})}
\label{eq:weight}
\end{eqnarray}
which means that the weighting function for the stacked observation is the sum of the weighting
function of the individual observations, weighted by the relative contribution of the model in the
individual observations.
The obvious problem is that the relative contribution of the model in the individual observations
is in not available in a stacked analysis.


%%%%%%%%%%%%%%%%%%%%%%%%%%%%%%%%%%%%%%%%%%%%%%%%%%%
% Ignoring partial overlap
%%%%%%%%%%%%%%%%%%%%%%%%%%%%%%%%%%%%%%%%%%%%%%%%%%%
\section{Ignoring partial overlap}

Suppose we can ignore the effects of partial overlap of the RoI with counts cube bins (we could
accomplish this in principle by redefining the bin boundaries so that there is not partial overlap
problem).
In that case the window functions of individual observations $w_i(\vec{d})$ take the values $0$
and $1$.
We can then distinguish four cases for our example of stacking two observations:
\begin{equation}
w(\vec{d}) = \left \{
   \begin{array}{c l}
      \displaystyle
      1 & w_1(\vec{d})=1, w_2(\vec{d})=1 \\
      \\
     \displaystyle
      \frac{M_1(\vec{d})}{M_1(\vec{d})+M_2(\vec{d})} & w_1(\vec{d})=1, w_2(\vec{d})=0 \\
      \\
     \displaystyle
      \frac{M_2(\vec{d})}{M_1(\vec{d})+M_2(\vec{d})} & w_1(\vec{d})=0, w_2(\vec{d})=1 \\
      \\
     \displaystyle
      0 & w_1(\vec{d})=0, w_2(\vec{d})=0
   \end{array}
   \right .
\label{eq:cases}
\end{equation}
For all data space coordinates that fall in both RoIs the stacked weighting function is 1, while for all
coordinates that fall outside both RoIs the stacked weighting function is 0.
In case that the data space coordinate $\vec{d}$ falls in the RoI of the first observation but outside the 
RoI of the second observation the weighting function becomes
\begin{equation}
w(\vec{d}) = \frac{M_1(\vec{d})}{M_1(\vec{d})+M_2(\vec{d})}
\end{equation}
which is smaller than 1 for all coordinates $\vec{d}$ for which $M_2(\vec{d}) > 0$.
These coordinates correspond to zones that fall {\bf outside the RoI of the second observation},
hence the reduction of the weighting function is due to the spill over of events from the second
observation coming from outside its RoI into the stacked observation.

A solution to this problem is to avoid the spill over, and to set $M_2(\vec{d}) = 0$ for all
$\vec{d}$ outside the RoI.
In other terms
\begin{equation}
\tilde{M_i}(\vec{d}) = w_i(\vec{d}) \times M_i(\vec{d})
\end{equation}
The issue here is that this cannot be done in an exact way for sky models, since they are defined 
in the physics space and not the data space, hence PSF blurring or energy redistribution cannot
be taken into account properly.
It nevertheless is probably still the best (maybe the only?) solution to limit the effect of the spill
over problem.


%%%%%%%%%%%%%%%%%%%%%%%%%%%%%%%%%%%%%%%%%%%%%%%%%%%
% Taking into account partial overlap
%%%%%%%%%%%%%%%%%%%%%%%%%%%%%%%%%%%%%%%%%%%%%%%%%%%
\section{Taking into account partial overlap}

In case we want to take into account the partial overlap of counts cube bins with the RoIs we
need to compute Eq.~(\ref{eq:weight}).
In general, the model can be factorised into 
\begin{equation}
M_i(\vec{d}) = R_i(\vec{d}) \times \tau_i
\end{equation}
where $R_i(\vec{d})$ is an event rate and $\tau_i$ is the livetime of the observation $i$.
Equation \ref{eq:weight} can then be rewritten as
\begin{equation}
w(\vec{d}) =
w_1(\vec{d}) \frac{R_1(\vec{d}) \times \tau_1}
{R_1(\vec{d}) \times \tau_1 + R_2(\vec{d}) \times \tau_2} +
w_2(\vec{d}) \frac{R_2(\vec{d}) \times \tau_2}
{R_1(\vec{d}) \times \tau_1 + R_2(\vec{d}) \times \tau_2}
\label{eq:overlap}
\end{equation}
We can still apply the logic of setting $R_i(\vec{d}) = 0$ outside the RoI, but there will be bins
(specifically in energy) for which the RoIs of both observations will have partial overlap, and we
need to compute Eq.~(\ref{eq:overlap}) explicitly.

For this we may consider some special cases.
The first is that $R_1(\vec{d}) \approx R_2(\vec{d})$ which would be the case if both observations
share roughly the same pointing and have the same energy threshold.
We can then drop the $R_i(\vec{d})$ from Eq.~(\ref{eq:overlap}) and get
\begin{equation}
w(\vec{d}) \approx w_1(\vec{d}) \frac{\tau_1}{\tau_1 + \tau_2} + w_2(\vec{d}) \frac{\tau_2}{\tau_1 + \tau_2}
\end{equation}
hence the weighting function for the stacked analysis is the sum of the weighting functions of the
observations, weighted by their respective livetimes.

Another special case is that there is negligible overlap between both observations, hence
$R_1(\vec{d}) \approx 0$ where $R_2(\vec{d})$ is non-zero, and vice versa.
Then all fractions in Eq.~(\ref{eq:overlap}) become unity and we get
\begin{equation}
w(\vec{d}) \approx w_1(\vec{d}) + w_2(\vec{d}) \, .
\end{equation}

We can also consider the special case $R_1(\vec{d}) \times \tau_1 \gg R_2(\vec{d}) \times \tau_2$ 
for which Eq.~(\ref{eq:overlap})
is approximated by
\begin{equation}
w(\vec{d}) \approx w_1(\vec{d}) + w_2(\vec{d}) \frac{R_2(\vec{d}) \times \tau_2}{R_1(\vec{d}) \times \tau_1}
\approx w_1(\vec{d}) \, .
\end{equation}

This illustrates that the weighting function taking into account partial overlaps can take different forms
dependent on the relative weights of the $R_i(\vec{d})$ or $M_i(\vec{d})$, and there is no obvious way
of how to do this properly without knowing the individual $M_i(\vec{d})$ or making an assumption about
them.

\end{document}

