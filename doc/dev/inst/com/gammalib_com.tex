%%%%%%%%%%%%%%%%%%%%%%%%%%%%%%%%%%%%%%%%%%%%%%%%%%%
% GammaLib COMPTEL Interface
%%%%%%%%%%%%%%%%%%%%%%%%%%%%%%%%%%%%%%%%%%%%%%%%%%%

%%%%%%%%%%%%%%%%%%%%%%%%%%%%%%%%%%%%%%%%%%%%%%%%%%%
% Definitions for manual package
%%%%%%%%%%%%%%%%%%%%%%%%%%%%%%%%%%%%%%%%%%%%%%%%%%%
\newcommand{\task}{\mbox{GammaLib}}
\newcommand{\this}{\mbox{\tt \task}}
\newcommand{\shorttype}{\mbox{COMPTEL}}
\newcommand{\doctype}{\mbox{COMPTEL interface}}
\newcommand{\version}{\mbox{draft}}
\newcommand{\calendar}{\mbox{17 January 2021}}
\newcommand{\auth}{\mbox{J\"urgen Kn\"odlseder}}
\newcommand{\approv}{\mbox{J\"urgen Kn\"odlseder}}
\def\sun{\hbox{$\odot$}}


%%%%%%%%%%%%%%%%%%%%%%%%%%%%%%%%%%%%%%%%%%%%%%%%%%%
% Document definition
%%%%%%%%%%%%%%%%%%%%%%%%%%%%%%%%%%%%%%%%%%%%%%%%%%%
\documentclass{article}[12pt,a4]
\usepackage{epsfig}
\usepackage{../manual}


%%%%%%%%%%%%%%%%%%%%%%%%%%%%%%%%%%%%%%%%%%%%%%%%%%%
% Begin of document body
%%%%%%%%%%%%%%%%%%%%%%%%%%%%%%%%%%%%%%%%%%%%%%%%%%%
\begin{document}
\frontpage


%%%%%%%%%%%%%%%%%%%%%%%%%%%%%%%%%%%%%%%%%%%%%%%%%%%
% Introduction
%%%%%%%%%%%%%%%%%%%%%%%%%%%%%%%%%%%%%%%%%%%%%%%%%%%
\section{Introduction}

The present document describes the COMPTEL specific interface that is implemented in \this.


%%%%%%%%%%%%%%%%%%%%%%%%%%%%%%%%%%%%%%%%%%%%%%%%%%%
% Definitions
%%%%%%%%%%%%%%%%%%%%%%%%%%%%%%%%%%%%%%%%%%%%%%%%%%%
\section{Definitions}

%%%%%%%%%%%%%%%%%%%%%%%%%%%%%%%%%%%%%%%%%%%%%%%%%%%
\subsection{Time}

COMPTEL times are defined by two integers: TJD (Truncated Julian Days) and tics (1/8 ms).
COMPTEL times are given in UTC, and TJD:tics = 8393:0 converts into 1991-05-17T00:00:00 UTC
(see COM-RP-UNH-DRG-037). 

At 8798:28800000 (1992-06-25T01:00:00) the CGRO clock was corrected, before that date the
CGRO clock was 2.042144 seconds too fast.

Conversion of COMPTEL times to a {\tt GTime} object is implemented by the {\tt gammalib::com\_time()}
function. The function applies the CGRO clock correction before 8798:28800000 and stores the
corrected time as UTC time. Specifically
\begin{equation}
{\rm MJD} = {\rm TJD} + 40000 + {\rm tics} \times \frac{0.000125}{86400}
\end{equation}
and for ${\rm TJD} < 8798$ or ${\rm TJD} = 8798$ and ${\rm tics} < 28800000$:
\begin{equation}
{\rm MJD}_{\rm corr} = {\rm MJD} - \frac{2.042144}{86400}
\end{equation}


%%%%%%%%%%%%%%%%%%%%%%%%%%%%%%%%%%%%%%%%%%%%%%%%%%%
% Pulsar analysis
%%%%%%%%%%%%%%%%%%%%%%%%%%%%%%%%%%%%%%%%%%%%%%%%%%%
\section{Pulsar}

%%%%%%%%%%%%%%%%%%%%%%%%%%%%%%%%%%%%%%%%%%%%%%%%%%%
\subsection{Notes of COMPASS code}

%%%%%%%%%%%%%%%%%%%%%%%%%%%%%%%%%%%%%%%%%%%%%%%%%%%
\subsubsection{Barycentre correction}

A Barycentre correction is applied to the event times. The formulae were extracted from the COMPASS
code {\tt pulphi01.pulphi.f}.

{\tt pulphi01.pulphi.f} reads the pulsar ephemeris from a {\tt PPT} file using the {\tt ppptrd.f} function. Specifically
the following parameters are read
\begin{itemize}
\item $t_0$ is the reference epoch of the pulsar ephemeris (specified as TJD and tics);
\item $\nu_0$ is the frequency at the reference epoch;
\item $\dot{\nu}_0$ is the first frequency derivative at the reference epoch (specified in units of $10^{-15}$);
\item $\ddot{\nu}_0$ is the second frequency derivative at the reference epoch (specified in units of $10^{-20}$);
\item $\alpha$ is the Right Ascension of the pulsar;
\item $\delta$ is the Declination of the pulsar;
\item $\nu_{\rm obs}$ is the observed frequency at radio;
\item $DM$ is the radio Dispersion measure;
\item $\Phi_0$ is the zero phase defining main radio peak.
\end{itemize}

The function computes then the centre time of the observation using
\begin{equation}
t_{\rm obs} = \frac{1}{2} (t_{\rm start} + t_{\rm end})
\end{equation}
where $t_{\rm start}$ and $t_{\rm end}$ are the start and stop time of the EVP dataset of
interest. 
Using the {\tt pula32.f} function, the pulsar frequency and frequency derivative as well as the zero phase
are extrapolated to this centre time using
\begin{equation}
\nu = \nu_0 + \dot{\nu}_0 (t_{\rm obs} - t_0)  + \frac{\ddot{\nu}_0}{2} (t_{\rm obs} - t_0)^2 ,
\end{equation}
\begin{equation}
\dot{\nu} = \dot{\nu}_0 + \ddot{\nu}_0 (t_{\rm obs} - t_0) ,
\end{equation}
and
\begin{equation}
\Phi = \Phi_0 + \nu_0 (t_{\rm obs} - t_0) + \frac{\dot{\nu}_0}{2} (t_{\rm obs} - t_0)^2 + \frac{\ddot{\nu}_0}{6} (t_{\rm obs} - t_0)^3
\label{eq:phase-ephemeris}
\end{equation}
with $\Phi$ put in the interval $[0,1]$.
Note that the second frequency derivative is not updated, as it would required a third frequency derivative for
that.

The subroutine then retrieves the Solar System Barycentre to CGRO spacecraft vector $\vec{s}$ based on the
information provided in the {\tt BVC} file.
Based on the event time $t_{\rm utc}$, given in the UTC time system, the routine uses the {\tt pulbcv.f} routine
to perform a linear interpolation between the two vectors $\vec{s}(i)$ and $\vec{s}(i+1)$ that enclose the event
time using
\begin{equation}
\vec{s} = (\vec{s}(i+1) - \vec{s}(i)) \frac{t_{\rm utc} - t(i) }{t(i+1) - t(i)} + \vec{s}(i)
\end{equation}
where $t(i)$ and $t(i+1)$ are the UTC times to which the vectors $i$ and $i+1$ correspond to.
If $t_{\rm utc}<t(0)$ then the first vector in the file is returned, if $t_{\rm utc}\ge t(N)$ then the last vector
is returned, where $N$ is the number of BVC vectors in the {\tt BVC} file.
Note that the {\tt BVC} file specifies $\vec{s}$ in units of micro light seconds.

The celestial vector of the pulsar position is then computed using
\begin{equation}
\vec{p} = \left\{ \cos \delta \cos \alpha, \cos \delta \sin \alpha, \sin \delta \right\}
\end{equation}
where $\alpha$ and $\delta$ are the Right Ascension and Declination of the pulsar.

The photon arrival time in the Solar System Barycentre $t_{\rm sbb}$, given in Barycentric Dynamic
Time (TDB), is then computed from the event time $t_{\rm utc}$ in the subroutine {\tt pula04.f} using
\begin{equation}
t_{\rm sbb} = t_{\rm utc} + \Delta t_{\rm travel} - \Delta t_{\rm relativistic} + \Delta t_{\rm system}
\end{equation}
where
\begin{equation}
\Delta t_{\rm travel} = 10^{-6} \times \vec{s} \cdot \vec{p} 
\end{equation}
is the light travel time from the CGRO to the Solar System Barycentre along the pulsar direction,
\begin{equation}
\Delta t_{\rm relativistic} = -2 t_{\sun} \log \left( 1 + \frac{ \Delta t_{\rm travel} }{10^{-6} | \vec{s} |} \right)
\end{equation}
is the relativistic delay due to the Sun, also known as Shapiro delay, with 
$t_{\sun} \equiv G M_{\sun} c^{-3} = 4.9254909 \times 10^{-6}$ s 
being half the Schwarzschild radius of the Sun divided by the speed of light, and
\begin{equation}
\Delta t_{\rm system} = N_{\rm leaps} + 32.184 + \Delta t_{TT \to TBB}
\end{equation}
converts from UTC to the TDB time system, with $N_{\rm leaps}$ being the number of leap seconds.
In all equations, time is expressed in units of seconds.

The pulse phase $\Phi_{\rm COMPTEL}$ for COMPTEL is then evaluated using
\begin{equation}
\Phi_{\rm COMPTEL} = \theta_0 + \Phi + \nu (t_{\rm sbb} - t_{\rm obs}) + \frac{\dot{\nu}}{2} (t_{\rm sbb} - t_{\rm obs})^2 + \frac{\ddot{\nu}_0}{6} (t_{\rm sbb} - t_{\rm obs})^3
\label{eq:phase-comptel}
\end{equation}
with $\Phi_{\rm COMPTEL}$ put in the interval $[0,1]$, where 
\begin{equation}
\theta_0 = \frac{-DM \times 10^{4}}{2.41 \times \nu_{\rm obs}^2}
\end{equation}
is the phase shift of radio pulsars due to the dispersion in the Interstellar Medium, with $DM$ in units of g/cm$^3$,
and $\theta_0=0$ for infinite frequency pulsar ephemerides.

Figure \ref{fig:pulsar-phase} compares the phase computation using Eq.~(\ref{eq:phase-comptel}) to the one using
Eq.~(\ref{eq:phase-ephemeris}), i.e.~without passing through an intermediate reference time $t_{\rm obs}$.
Obviously the phase difference is negligible, and the computation, implemented in {\tt pula32.f}, of an intermediate
reference time can be avoided.

%%%%%%%%%%%%%%%%%%%%%%%%%%%%%%%%%%%%%%%%%%%%%%%%%%%
\begin{figure}[!t]
\center
\epsfig{figure=com-pulsar-phase.eps, width=12cm}
\caption{Maximum phase difference between computing the phase via an intermediate 
reference time $t_{\rm obs}$ using Eq.~(\ref{eq:phase-comptel}) or via using the ephemeris 
reference time $t_0$ using Eq.~(\ref{eq:phase-ephemeris}). The phase difference is negligible.}
\label{fig:pulsar-phase}
\end{figure}
%%%%%%%%%%%%%%%%%%%%%%%%%%%%%%%%%%%%%%%%%%%%%%%%%%%


%%%%%%%%%%%%%%%%%%%%%%%%%%%%%%%%%%%%%%%%%%%%%%%%%%%
\subsubsection{PPT files}

The {\tt PPT} files are written by the COMPASS task {\tt pulcpp.f}. It writes a number of parameters,
such as pulsar position, frequency and derivatives, and computes in particular the zero phase $\Phi_0$
defining the main radio peak using
\begin{equation}
\Phi_0 = - \left( \nu_0 \Delta t + \frac{\dot{\nu}_0}{2} \Delta t^2 + \frac{\ddot{\nu}_0}{6} \Delta t^3 \right)
\end{equation}
where
$\Phi_0$ is put in the interval $[-1,0]$.
Here $\nu_0$ is the radio frequency, $\dot{\nu}_0$ the first frequency derivative and $\ddot{\nu}_0$
is the second frequency derivative.
The time $\Delta t$ since the reference time is computed using
\begin{equation}
\Delta t = f + \Delta t_{\rm cor} + \Delta t_{\rm delay}
\end{equation}
where
$f$ is the fractional part of the reference time in a geocentric reference frame (t0geo in the psrtime files),
$\Delta t_{\rm cor}$ is a correction parameter that can be specified by the user, and
\begin{equation}
\Delta t_{\rm delay} = \Delta t_{\rm travel} - \Delta t_{\rm relativistic} + \Delta t_{\rm system}
\end{equation}
is the time delay, with
\begin{equation}
\Delta t_{\rm travel} = \vec{r} \cdot \vec{p} 
\end{equation}
being the light travel time from the geocentre to the Solar System Barycentre along the pulsar direction,
$\vec{r}$ is the vector from the Solar System Barycentre to the geocentre in light seconds,
\begin{equation}
\Delta t_{\rm relativistic} = -2 t_{\sun} \log \left( 1 + \frac{ \Delta t_{\rm travel} }{| \vec{r} |} \right)
\end{equation}
is the relativistic delay due to the Sun, also known as Shapiro delay, with 
$t_{\sun} \equiv G M_{\sun} c^{-3} = 4.9254909 \times 10^{-6}$ s 
being half the Schwarzschild radius of the Sun divided by the speed of light, and
\begin{equation}
\Delta t_{\rm system} = N_{\rm leaps} + 32.184 + \Delta t_{TT \to TBB}
\end{equation}
converts from UTC to the TDB time system, with $N_{\rm leaps}$ being the number of leap seconds
(as a side note, it seems that in COMPASS the {\tt NLEAP} parameter counts the leap seconds from
1972 on, and since 10 leap seconds were added in 1972 the code uses internally {\tt NLEAP + 10}).
In all equations, time is expressed in units of seconds.
Note that $\vec{r}$ and $\Delta t_{\rm system}$ are computed using the function {\tt bvceph.f}.

\end{document}
