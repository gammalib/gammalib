%%%%%%%%%%%%%%%%%%%%%%%%%%%%%%%%%%%%%%%%%%%%%%%%%%%
% GammaLib CTA Interface
%%%%%%%%%%%%%%%%%%%%%%%%%%%%%%%%%%%%%%%%%%%%%%%%%%%

%%%%%%%%%%%%%%%%%%%%%%%%%%%%%%%%%%%%%%%%%%%%%%%%%%%
% Definitions for manual package
%%%%%%%%%%%%%%%%%%%%%%%%%%%%%%%%%%%%%%%%%%%%%%%%%%%
\newcommand{\task}{\mbox{GammaLib}}
\newcommand{\this}{\mbox{\tt \task}}
\newcommand{\shorttype}{\mbox{CTA}}
\newcommand{\doctype}{\mbox{CTA interface}}
\newcommand{\version}{\mbox{draft}}
\newcommand{\calendar}{\mbox{23 September 2020}}
\newcommand{\auth}{\mbox{J\"urgen Kn\"odlseder}}
\newcommand{\approv}{\mbox{J\"urgen Kn\"odlseder}}


%%%%%%%%%%%%%%%%%%%%%%%%%%%%%%%%%%%%%%%%%%%%%%%%%%%
% Document definition
%%%%%%%%%%%%%%%%%%%%%%%%%%%%%%%%%%%%%%%%%%%%%%%%%%%
\documentclass{article}[12pt,a4]
\usepackage{epsfig}
\usepackage{../manual}


%%%%%%%%%%%%%%%%%%%%%%%%%%%%%%%%%%%%%%%%%%%%%%%%%%%
% Begin of document body
%%%%%%%%%%%%%%%%%%%%%%%%%%%%%%%%%%%%%%%%%%%%%%%%%%%
\begin{document}
\frontpage


%%%%%%%%%%%%%%%%%%%%%%%%%%%%%%%%%%%%%%%%%%%%%%%%%%%
% Introduction
%%%%%%%%%%%%%%%%%%%%%%%%%%%%%%%%%%%%%%%%%%%%%%%%%%%
\section{Introduction}

The present document describes the CTA specific interface that is implemented in \this.



%%%%%%%%%%%%%%%%%%%%%%%%%%%%%%%%%%%%%%%%%%%%%%%%%%%
% Instrument response
%%%%%%%%%%%%%%%%%%%%%%%%%%%%%%%%%%%%%%%%%%%%%%%%%%%
\section{Instrument response}

%%%%%%%%%%%%%%%%%%%%%%%%%%%%%%%%%%%%%%%%%%%%%%%%%%%
\subsection{Response for stacked analysis}

%%%%%%%%%%%%%%%%%%%%%%%%%%%%%%%%%%%%%%%%%%%%%%%%%%%
\subsubsection{Radial models}

A radial model is defined by a function $f(\theta)$ where $\theta$ is the angular distance between
the photon direction $(\alpha,\beta)$ and the model centre $(\alpha_0,\beta_0)$, given by
\begin{equation}
\theta = \arccos{ \left( \sin \beta \sin \beta_0 + \cos \beta \cos \beta_0 \cos (\alpha-\alpha_0) \right) } .
\end{equation}
The spatial integration of the model is performed in a spherical coordinate system that is centred on
the reconstructed event direction $(\hat{\alpha},\hat{\beta})$, spanned by the offset angle
\begin{equation}
\delta = \arccos{ \left( \sin \beta \sin \hat{\beta} + \cos \beta \cos \hat{\beta} \cos (\alpha-\hat{\alpha}) \right) } ,
\end{equation}
which is the angular distance between true and reconstructed event direction, and
\begin{equation}
\varphi =
\arctan \left( \frac{\sin(\alpha-\hat{\alpha})}{\cos \hat{\beta} \tan \beta - \sin \hat{\beta} \cos(\alpha-\hat{\alpha})} \right) -
\arctan \left( \frac{\sin(\alpha_0-\hat{\alpha})}{\cos \hat{\beta} \tan \beta_0 - \sin \hat{\beta} \cos(\alpha_0-\hat{\alpha})} \right)
\end{equation}
which is the azimuth angle around the reconstructed event direction $(\hat{\alpha},\hat{\beta})$,
measured from a line that connects $(\hat{\alpha},\hat{\beta})$ and $(\alpha_0,\beta_0)$.
In this system, the radial model argument $\theta$ is given by
\begin{equation}
\theta = \arccos{ \left( \cos \delta \cos \zeta + \sin \delta \sin \zeta \cos \varphi \right) }
\end{equation}
where
$\zeta$ is the angular distance between the reconstructed event direction $(\hat{\alpha},\hat{\beta})$ and
the model centre $(\alpha_0,\beta_0)$, given by
\begin{equation}
\zeta = \arccos{ \left( \sin \beta_0 \sin \hat{\beta} + \cos \beta_0 \cos \hat{\beta} \cos (\alpha_0-\hat{\alpha}) \right) } .
\end{equation}
Figure~\ref{fig:radial-model} illustrates the definition of the angles.

%%%%%%%%%%%%%%%%%%%%%%%%%%%%%%%%%%%%%%%%%%%%%%%%%%%
\begin{figure}[!t]
\center
\epsfig{figure=cta-radial-models.eps, width=10cm}
\caption{Angles for the computation of the response for stacked analysis. The red circle indicates
the zone where the model is valid, the blue circle indicates the zone where the PSF is positive. The
model integration is performed in a spherical system centred on the reconstructed event
direction $(\alpha_0,\beta_0)$ that is spanned by $\delta$ and $\varphi$.}
\label{fig:radial-model}
\end{figure}
%%%%%%%%%%%%%%%%%%%%%%%%%%%%%%%%%%%%%%%%%%%%%%%%%%%

The partial derivatives of $f(\theta)$ with respect to $\alpha_0$ and $\beta_0$ are given by
\begin{equation}
\frac{\partial f}{\partial \alpha_0} = \frac{\partial f}{\partial \theta} \frac{\partial \theta}{\partial \alpha_0}
\end{equation}
and
\begin{equation}
\frac{\partial f}{\partial \beta_0} = \frac{\partial f}{\partial \theta} \frac{\partial \theta}{\partial \beta_0}.
\end{equation}
For the general case,
\begin{equation}
\frac{\partial \theta}{\partial \alpha_0} = 
\frac{\partial \theta}{\partial \zeta} \frac{\partial \zeta}{\partial \alpha_0} +
\frac{\partial \theta}{\partial \varphi} \frac{\partial \varphi}{\partial \alpha_0}
\end{equation}
and
\begin{equation}
\frac{\partial \theta}{\partial \beta_0} =
\frac{\partial \theta}{\partial \zeta} \frac{\partial \zeta}{\partial \beta_0} +
\frac{\partial \theta}{\partial \varphi} \frac{\partial \varphi}{\partial \beta_0}
\end{equation}
with
\begin{equation}
\frac{\partial \theta}{\partial \zeta} = \frac{\cos \delta \sin \zeta - \sin \delta \cos \zeta \cos \varphi}{\sin \theta} ,
\end{equation}
\begin{equation}
\frac{\partial \theta}{\partial \varphi} = \frac{\sin \delta \sin \zeta \sin \varphi}{\sin \theta} ,
\end{equation}
\begin{equation}
\frac{\partial \zeta}{\partial \alpha_0}  = \frac{\cos \beta_0 \cos \hat{\beta} \sin(\alpha_0-\hat{\alpha})}{\sin \zeta} ,
\end{equation}
\begin{equation}
\frac{\partial \zeta}{\partial \beta_0}  = \frac{\sin \beta_0 \cos \hat{\beta} \cos(\alpha_0-\hat{\alpha}) - \cos \beta_0 \sin \hat{\beta}}{\sin \zeta} ,
\end{equation}
\begin{equation}
\frac{\partial \varphi}{\partial \alpha_0}  = \frac
{\sin \hat{\beta} - \cos(\alpha_0-\hat{\alpha}) \cos \hat{\beta} \tan \beta_0}
{\sin^2(\alpha_0-\hat{\alpha}) + (\cos \hat{\beta} \tan \beta_0 - \sin \hat{\beta} \cos(\alpha_0-\hat{\alpha}))^2} ,
\end{equation}
and
\begin{equation}
\frac{\partial \varphi}{\partial \beta_0}  = \frac
{\sin(\alpha_0-\hat{\alpha}) \cos \hat{\beta} (1 + \tan^2 \beta_0)}
{\sin^2(\alpha_0-\hat{\alpha}) + (\cos \hat{\beta} \tan \beta_0 - \sin \hat{\beta} \cos(\alpha_0-\hat{\alpha}))^2} ,
\end{equation}
Several special cases need to be considered.
For $\theta=0$
\begin{equation}
\frac{\partial \theta}{\partial \alpha_0} = \frac{\partial \theta}{\partial \beta_0} = 0
\end{equation}
For $\zeta=0$ ...
\begin{equation}
\frac{\partial \theta}{\partial \alpha_0} = ?
\end{equation}
and
\begin{equation}
\frac{\partial \theta}{\partial \beta_0} = - \cos \varphi
\end{equation}

%%%%%%%%%%%%%%%%%%%%%%%%%%%%%%%%%%%%%%%%%%%%%%%%%%%
\paragraph{Radial disk}

The radial disk function $f(\theta)$ is given by
\begin{equation}
f(\theta) = \left \{
   \begin{array}{l l}
     \displaystyle \frac{1}{2 \pi (1 - \cos \theta_0)} & \mbox{if $\theta \le \theta_0$} \\ \\
     \displaystyle 0 & \mbox{if $\theta > \theta_0$}
   \end{array}
   \right .
\end{equation}
with the partial derivatives
\begin{equation}
\frac{\partial f}{\partial \theta}  = \delta(\theta - \theta_0)
\end{equation}
and
\begin{equation}
\frac{\partial f}{\partial \theta_0} = \left \{
   \begin{array}{l l}
     \displaystyle -2 \pi f^2(\theta) \sin \theta_0 & \mbox{if $\theta \le \theta_0$} \\ \\
     \displaystyle 0 & \mbox{if $\theta > \theta_0$}
   \end{array}
   \right .
\end{equation}

%%%%%%%%%%%%%%%%%%%%%%%%%%%%%%%%%%%%%%%%%%%%%%%%%%%
\paragraph{Radial ring}

The radial ring function $f(\theta)$ is given by
\begin{equation}
f(\theta) = \left \{
   \begin{array}{l l}
     \displaystyle 0 & \mbox{if $\theta < \theta_{\rm in}$} \\ \\
     \displaystyle \frac{1}{2 \pi (\cos \theta_{\rm in} - \cos \theta_{\rm out})} & \mbox{if $\theta \ge \theta_{\rm in}$ and $\theta \le \theta_{\rm out}$} \\ \\
     \displaystyle 0 & \mbox{if $\theta > \theta_{\rm out}$}
   \end{array}
   \right .
\end{equation}
with the partial derivatives
\begin{equation}
\frac{\partial f}{\partial \theta}  = ?
\end{equation}
and
\begin{equation}
\frac{\partial f}{\partial \theta_{\rm in}}  = ?
\end{equation}
and
\begin{equation}
\frac{\partial f}{\partial \theta_{\rm out}}  = ? .
\end{equation}

%%%%%%%%%%%%%%%%%%%%%%%%%%%%%%%%%%%%%%%%%%%%%%%%%%%
\paragraph{Radial Gaussian}

The radial Gaussian function $f(\theta)$ is given by
\begin{equation}
f(\theta) = \frac{1}{2 \pi \sigma^2} \exp \left( -\frac{1}{2} \frac{\theta^2}{\sigma^2} \right)
\end{equation}
with the partial derivatives
\begin{equation}
\frac{\partial f}{\partial \theta}  = - \frac{\theta}{\sigma^2} f(\theta)
\end{equation}
and
\begin{equation}
\frac{\partial f}{\partial \sigma}  = \left( \frac{\theta^2}{\sigma^3} - \frac{2}{\sigma} \right) f(\theta) .
\end{equation}

%%%%%%%%%%%%%%%%%%%%%%%%%%%%%%%%%%%%%%%%%%%%%%%%%%%
\paragraph{Radial shell}

The shell function on a sphere is given by
\begin{equation}
f(\theta) = f_0 \left \{
   \begin{array}{l l}
      \displaystyle
      \sqrt{ \sin^2 \theta_{\rm out} - \sin^2 \theta } - \sqrt{ \sin^2 \theta_{\rm in} - \sin^2 \theta }
      & \mbox{if $\theta \le \theta_{\rm in}$} \\
      \\
     \displaystyle
      \sqrt{ \sin^2 \theta_{\rm out} - \sin^2 \theta }
      & \mbox{if $\theta_{\rm in} < \theta \le \theta_{\rm out}$} \\
      \\
     \displaystyle
     0 & \mbox{if $\theta > \theta_{\rm out}$}
   \end{array}
   \right .
\end{equation}
The normalization constant $f_0$ is determined by
\begin{equation}
2 \pi \int_0^{\pi/2} f(\theta) \sin \theta \, {\rm d}\theta = 1
\end{equation}

\end{document}
