%%%%%%%%%%%%%%%%%%%%%%%%%%%%%%%%%%%%%%%%%%%%%%%%%%%
% GammaLib Mathematical Implementation
%%%%%%%%%%%%%%%%%%%%%%%%%%%%%%%%%%%%%%%%%%%%%%%%%%%

%%%%%%%%%%%%%%%%%%%%%%%%%%%%%%%%%%%%%%%%%%%%%%%%%%%
% Definitions for manual package
%%%%%%%%%%%%%%%%%%%%%%%%%%%%%%%%%%%%%%%%%%%%%%%%%%%
\newcommand{\task}{\mbox{GammaLib}}
\newcommand{\this}{\mbox{\tt \task}}
\newcommand{\shorttype}{\mbox{Maths}}
\newcommand{\doctype}{\mbox{Mathematical Implementation}}
\newcommand{\version}{\mbox{draft}}
\newcommand{\calendar}{\mbox{2 January 2011}}
\newcommand{\auth}{\mbox{J\"urgen Kn\"odlseder}}
\newcommand{\approv}{\mbox{J\"urgen Kn\"odlseder}}


%%%%%%%%%%%%%%%%%%%%%%%%%%%%%%%%%%%%%%%%%%%%%%%%%%%
% Document definition
%%%%%%%%%%%%%%%%%%%%%%%%%%%%%%%%%%%%%%%%%%%%%%%%%%%
\documentclass{article}[12pt,a4]
\usepackage{epsfig}
\usepackage{manual}


%%%%%%%%%%%%%%%%%%%%%%%%%%%%%%%%%%%%%%%%%%%%%%%%%%%
% Begin of document body
%%%%%%%%%%%%%%%%%%%%%%%%%%%%%%%%%%%%%%%%%%%%%%%%%%%
\begin{document}
\frontpage


%%%%%%%%%%%%%%%%%%%%%%%%%%%%%%%%%%%%%%%%%%%%%%%%%%%
% Introduction
%%%%%%%%%%%%%%%%%%%%%%%%%%%%%%%%%%%%%%%%%%%%%%%%%%%
\section{Introduction}


%%%%%%%%%%%%%%%%%%%%%%%%%%%%%%%%%%%%%%%%%%%%%%%%%%%
% Spectral models
%%%%%%%%%%%%%%%%%%%%%%%%%%%%%%%%%%%%%%%%%%%%%%%%%%%
\section{Spectral models}

%%%%%%%%%%%%%%%%%%%%%%%%%%%%%%%%%%%%%%%%%%%%%%%%%%%
\subsection{PowerLaw}

The power law spectral model is defined by
\begin{equation}
I(E) = k \left( \frac{E}{p} \right)^{\gamma}
\end{equation}
where
\begin{itemize}
\item $k$ is the normalization of the power law (units: ph cm$^{-2}$ s$^{-1}$ MeV$^{-1}$),
\item $p$ is the pivot energy (units: MeV), and
\item $\gamma$ is the spectral index (which is usually negative).
\end{itemize}

Each of the 3 parameters is factorised into a scaling factor and a value, i.e.
$k=k_s k_v$, $p=p_s p_v$, and $\gamma = \gamma_s \gamma_v$.
The {\tt GModelSpectralPlaw::eval\_gradients} returns the gradients with
respect to the parameter value of the factorisation.
Note that for any parameter $a=a_s a_v$:
\begin{equation}
\frac{\delta I}{\delta a_v} = \frac{\delta I}{\delta a} \frac{\delta a}{\delta a_v} =
  \frac{\delta I}{\delta a} a_s
\end{equation}
The parameter value gradients for the power law are given by
\begin{eqnarray}
\frac{\delta I}{\delta k_v} & = & 
  k_s \left( \frac{E}{p} \right)^{\gamma} = \frac{I(E)}{k_v} \\
\frac{\delta I}{\delta p_v} & = & 
  -\frac{\gamma}{p_v} \, k \left( \frac{E}{p} \right)^{\gamma} =
  -\frac{\gamma}{p_v} \, I(E) \\
\frac{\delta I}{\delta \gamma_v} & = &
  \gamma_s \ln \left( \frac{E}{p} \right) k \left( \frac{E}{p} \right)^{\gamma} =
  \gamma_s \ln \left( \frac{E}{p} \right) I(E)
\end{eqnarray}


%%%%%%%%%%%%%%%%%%%%%%%%%%%%%%%%%%%%%%%%%%%%%%%%%%%
\subsection{PowerLaw2}

This flavour of the power law spectral model uses the integral flux $f$ within the
energy range $E_{\rm min}$ and $E_{\rm max}$ as free parameter instead of
the normalization $k$.
The integral flux $f$ is given by
\begin{equation}
f = \int_{E_{\rm min}}^{E_{\rm max}} I(E) {\rm d}E
\label{eq:fluxint}
\end{equation}
The power law model is then defined by
\begin{equation}
I(E) = \tilde{k} \, E^{\gamma}
\end{equation}
where
\begin{equation}
   \tilde{k} = \left \{
   \begin{array}{l l}
      \displaystyle
      \frac{f}{\ln E_{\rm max} - \ln E_{\rm min}} 
        & \mbox{if $\gamma = -1$} \\
     \\
     \displaystyle
      \frac{f (1+\gamma)}{E_{\rm max}^{\gamma+1}-E_{\rm min}^{\gamma+1}} 
        & \mbox{else}
   \end{array}
   \right .
\end{equation}
is obtained from Eq.~(\ref{eq:fluxint}).

Each of the 4 parameters is factorised into a scaling factor and a value, e.g.
$f=f_s f_v$ and $\gamma=\gamma_s \gamma_v$.
It is assumed that $E_{\rm min}$ and $E_{\rm max}$ are fixed parameters,
and {\tt GModelSpectralPlaw2::eval\_gradients} returns valid gradients only for 
$f_v$ and $\gamma_v$.

The flux value gradient $f_v$ is given by
\begin{equation}
\frac{\delta I}{\delta f_v} = 
  \frac{\delta I}{\delta f} \frac{\delta f}{\delta f_v} = 
  \frac{\delta I}{\delta f} f_s = 
  \frac{\delta \tilde{k}}{\delta f} \, E^{\gamma} f_s = 
  \frac{\tilde{k}}{f} \, E^{\gamma} f_s = 
  \frac{I(E)}{f} f_s =
  \frac{I(E)}{f_v}
\end{equation}

The index value gradient $\gamma_v$ is given by
\begin{equation}
\frac{\delta I}{\delta \gamma_v} =
  \frac{\delta I}{\delta \gamma} \frac{\delta \gamma}{\delta \gamma_v} = 
  \frac{\delta I}{\delta \gamma} \gamma_s = 
  \left( \frac{\delta \tilde{k}}{\delta \gamma} \, E^{\gamma} + \tilde{k} \, E^{\gamma} \ln E \right)  \gamma_s =
  \left( \frac{1}{\tilde{k}} \frac{\delta \tilde{k}}{\delta \gamma} + \ln E \right) \tilde{k} E^{\gamma} \gamma_s =
  \left( \frac{1}{\tilde{k}} \frac{\delta \tilde{k}}{\delta \gamma} + \ln E \right) I(E) \gamma_s
\end{equation}
where
\begin{equation}
   \frac{\delta \tilde{k}}{\delta \gamma} = \left \{
   \begin{array}{l l}
     \displaystyle
     0 & \mbox{if $\gamma = -1$} \\
     \\
     \\
     \displaystyle
      \frac{f \left( E_{\rm max}^{\gamma+1}-E_{\rm min}^{\gamma+1} \right) -
              f (1+\gamma) \left( E_{\rm max}^{\gamma+1} \ln E_{\rm max} - 
                                      E_{\rm min}^{\gamma+1} \ln E_{\rm min} \right)}
              {\left( E_{\rm max}^{\gamma+1}-E_{\rm min}^{\gamma+1} \right)^2}
         & \mbox{else}
   \end{array}
   \right .
\end{equation}
Note that for $\gamma \ne -1$
\begin{equation}
\frac{1}{\tilde{k}} \frac{\delta \tilde{k}}{\delta \gamma} =
  \frac{1}{1+\gamma} -
  \frac{\left( E_{\rm max}^{\gamma+1} \ln E_{\rm max} - 
                    E_{\rm min}^{\gamma+1} \ln E_{\rm min} \right)}
          {\left( E_{\rm max}^{\gamma+1}-E_{\rm min}^{\gamma+1} \right)}
\end{equation}

\end{document} 
