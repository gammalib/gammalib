%%%%%%%%%%%%%%%%%%%%%%%%%%%%%%%%%%%%%%%%%%%%%%%%%%%
% GammaLib Mathematical Implementation
%%%%%%%%%%%%%%%%%%%%%%%%%%%%%%%%%%%%%%%%%%%%%%%%%%%

%%%%%%%%%%%%%%%%%%%%%%%%%%%%%%%%%%%%%%%%%%%%%%%%%%%
% Definitions for manual package
%%%%%%%%%%%%%%%%%%%%%%%%%%%%%%%%%%%%%%%%%%%%%%%%%%%
\newcommand{\task}{\mbox{GammaLib}}
\newcommand{\this}{\mbox{\tt \task}}
\newcommand{\shorttype}{\mbox{Maths}}
\newcommand{\doctype}{\mbox{Mathematical Implementation}}
\newcommand{\version}{\mbox{draft}}
\newcommand{\calendar}{\mbox{23 February 2013}}
\newcommand{\auth}{\mbox{J\"urgen Kn\"odlseder}}
\newcommand{\approv}{\mbox{J\"urgen Kn\"odlseder}}


%%%%%%%%%%%%%%%%%%%%%%%%%%%%%%%%%%%%%%%%%%%%%%%%%%%
% Document definition
%%%%%%%%%%%%%%%%%%%%%%%%%%%%%%%%%%%%%%%%%%%%%%%%%%%
\documentclass{article}[12pt,a4]
\usepackage{epsfig}
\usepackage{manual}


%%%%%%%%%%%%%%%%%%%%%%%%%%%%%%%%%%%%%%%%%%%%%%%%%%%
% Begin of document body
%%%%%%%%%%%%%%%%%%%%%%%%%%%%%%%%%%%%%%%%%%%%%%%%%%%
\begin{document}
\frontpage


%%%%%%%%%%%%%%%%%%%%%%%%%%%%%%%%%%%%%%%%%%%%%%%%%%%
% Introduction
%%%%%%%%%%%%%%%%%%%%%%%%%%%%%%%%%%%%%%%%%%%%%%%%%%%
\section{Introduction}

This document provides the derivation of some formulae used in \this.


%%%%%%%%%%%%%%%%%%%%%%%%%%%%%%%%%%%%%%%%%%%%%%%%%%%
% Functions
%%%%%%%%%%%%%%%%%%%%%%%%%%%%%%%%%%%%%%%%%%%%%%%%%%%
\section{Functions}

%%%%%%%%%%%%%%%%%%%%%%%%%%%%%%%%%%%%%%%%%%%%%%%%%%%
\subsection{King profile}

The King Profile is a commonly used parametrisation of instrument PSFs in astronomy. 
It is radially symmetric and compared to a simple Gaussian, it allows longer tails in the distribution 
of events from a point source. 
The probability density function is defined as follows:
\begin{equation}
P(r|\sigma, \gamma) = \frac{1}{2 \pi \sigma^2} \left( 1 - \frac{1}{\gamma} \right)
\left( 1 + \frac{1}{2 \gamma} \frac{r^2}{\sigma^2} \right)^{-\gamma}
\end{equation}

To integrate the function we substitute
\begin{equation}
u = \frac{r^2}{2 \sigma^2}
\end{equation}
which gives
\begin{equation}
r dr = \sigma^2 du
\end{equation}
and hence
\begin{equation}
\frac{1}{2 \pi \sigma^2} \left( 1 - \frac{1}{\gamma} \right)
\left( 1 + \frac{1}{2 \gamma} \frac{r^2}{\sigma^2} \right)^{-\gamma} r dr =
\frac{1}{2 \pi \sigma^2} \left( 1 - \frac{1}{\gamma} \right)
\left( 1 + \frac{u}{\gamma} \right)^{-\gamma} \sigma^2 du
\end{equation}

The integral is thus given by
\begin{eqnarray}
\int_0^{r_{\rm max}} \int_0^{2 \pi} P(r|\sigma, \gamma) r dr d\phi & = &
\left( 1 - \frac{1}{\gamma} \right) \int_0^{u_{\rm max}} 
\left( 1 + \frac{u}{\gamma} \right)^{-\gamma} du \nonumber \\
& = & \left.
\left( 1 - \frac{1}{\gamma} \right)
\left(\frac{(\gamma + u) \left( \frac{ \gamma + u}{\gamma} \right)^{-\gamma}}{1-\gamma} \right) 
\right|_0^{u_{\rm max}} \nonumber \\
& = & \left.
\left( \frac{\gamma-1}{\gamma} \right)
\left(\frac{(\gamma + u) \left( \frac{ \gamma + u}{\gamma} \right)^{-\gamma}}{1-\gamma} \right) 
\right|_0^{u_{\rm max}} \nonumber \\
& = & \left.
-\left(\frac{(\gamma + u) \left( \frac{ \gamma + u}{\gamma} \right)^{-\gamma}}{\gamma} \right) 
\right|_0^{u_{\rm max}} \nonumber \\
& = & \left.
-\left( \frac{ \gamma + u}{\gamma} \right)^{1-\gamma}
\right|_0^{u_{\rm max}} \nonumber \\
& = & \left.
-\left( 1 + \frac{u}{\gamma} \right)^{1-\gamma}
\right|_0^{u_{\rm max}} \nonumber \\
& = & 
-\left( 1 + \frac{u_{\rm max}}{\gamma} \right)^{1-\gamma}
+1 \nonumber \\
& = & 1 - \left( 1 + \frac{u_{\rm max}}{\gamma} \right)^{1-\gamma}
\end{eqnarray}
where
\begin{equation}
u_{\rm max} = \frac{r_{\rm max}^2}{2 \sigma^2}
\end{equation}

To determine the value of $u_{\rm max}$ that corresponds to a given containment fraction $F$,
the equation
\begin{equation}
1 - \left( 1 + \frac{u_{\rm max}}{\gamma} \right)^{1-\gamma} = F
\end{equation}
has to be solved for $u_{\rm max}$, resulting in
\begin{equation}
u_{\rm max} = \left( \left( 1 - F \right)^{\frac{1}{1-\gamma}} - 1 \right) \gamma
\end{equation}
This can be converted into $r_{\rm max}$ using 
\begin{equation}
r_{\rm max} = \sigma \sqrt{ 2 u_{\rm max}}
\end{equation}



%%%%%%%%%%%%%%%%%%%%%%%%%%%%%%%%%%%%%%%%%%%%%%%%%%%
% Spatial models
%%%%%%%%%%%%%%%%%%%%%%%%%%%%%%%%%%%%%%%%%%%%%%%%%%%
\section{Spatial models}

%%%%%%%%%%%%%%%%%%%%%%%%%%%%%%%%%%%%%%%%%%%%%%%%%%%
\subsection{Radial Shell}

\subsubsection{Small angle approximation}

The shell function is a radial function $f(\theta)$, where $\theta$ is the angular separation
between shell centre and the actual location.
In the small angle approximation $\sin \theta \approx \theta$, the shell function is given
by
\begin{equation}
f(\theta) = f_0 \left \{
   \begin{array}{l l}
      \displaystyle
      \sqrt{ \theta_{\rm out}^2 - \theta^2 } - \sqrt{ \theta_{\rm in}^2 - \theta^2 }
      & \mbox{if $\theta \le \theta_{\rm in}$} \\
      \\
     \displaystyle
      \sqrt{ \theta_{\rm out}^2 - \theta^2 }
      & \mbox{if $\theta_{\rm in} < \theta \le \theta_{\rm out}$} \\
      \\
     \displaystyle
     0 & \mbox{if $\theta > \theta_{\rm out}$}
   \end{array}
   \right .
\end{equation}
is the radial function.
$f_0$ is a normalization constant that is determined by
\begin{equation}
2 \pi \int_0^{\pi/2} f(\theta) \theta \, {\rm d}\theta = 1
\end{equation}
in the small angle approximation.
Using
\begin{equation}
\int x \sqrt {a-x^2} \, {\rm d}x = -\frac{1}{3} (a-x^2)^{3/2}
\end{equation}
we obtain
\begin{eqnarray}
\frac{1}{f_0} & = &
  -\frac{2 \pi}{3} \left[ (\theta_{\rm out}^2 - \theta_{\rm in}^2)^{3/2} -
                               (\theta_{\rm in}^2 - \theta_{\rm in}^2)^{3/2} -
                               (\theta_{\rm out}^2)^{3/2} +
                               (\theta_{\rm in}^2)^{3/2} +
                               (\theta_{\rm out}^2 - \theta_{\rm out}^2)^{3/2} -
                               (\theta_{\rm out}^2 - \theta_{\rm in}^2)^{3/2} \right] \\
& = & -\frac{2 \pi}{3} \left[ (\theta_{\rm out}^2 - \theta_{\rm in}^2)^{3/2} -
                                        (\theta_{\rm out}^2)^{3/2} +
                                        (\theta_{\rm in}^2)^{3/2} -
                                        (\theta_{\rm out}^2 - \theta_{\rm in}^2)^{3/2} \right] \\
& = & -\frac{2 \pi}{3} \left[ -(\theta_{\rm out}^2)^{3/2} + (\theta_{\rm in}^2)^{3/2} \right] \\
& = & \frac{2 \pi}{3} \left[ \theta_{\rm out}^3 - \theta_{\rm in}^3 \right]
\end{eqnarray}


\subsubsection{Spherical formulation}
 
The shell function on a sphere is given by
\begin{equation}
f(\theta) = f_0 \left \{
   \begin{array}{l l}
      \displaystyle
      \sqrt{ \sin^2 \theta_{\rm out} - \sin^2 \theta } - \sqrt{ \sin^2 \theta_{\rm in} - \sin^2 \theta }
      & \mbox{if $\theta \le \theta_{\rm in}$} \\
      \\
     \displaystyle
      \sqrt{ \sin^2 \theta_{\rm out} - \sin^2 \theta }
      & \mbox{if $\theta_{\rm in} < \theta \le \theta_{\rm out}$} \\
      \\
     \displaystyle
     0 & \mbox{if $\theta > \theta_{\rm out}$}
   \end{array}
   \right .
\end{equation}
The normalization constant $f_0$ is determined by
\begin{equation}
2 \pi \int_0^{\pi/2} f(\theta) \sin \theta \, {\rm d}\theta = 1
\end{equation}
Using
\begin{equation}
\int \sin x \sqrt {a - \sin^2 x} \, {\rm d}x = 
-\frac{\cos x \sqrt{\cos 2x + 2a-1}}{2 \sqrt{2}}
-\frac{a-1}{2} \ln \left( \sqrt{2} \cos x + \sqrt{\cos 2x + 2a-1} \right)
\end{equation}
and
\begin{equation}
2 a = 2\sin^2 \theta_0 = 1-\cos 2 \theta_0
\end{equation}
we can write
\begin{equation}
\int \sin x \sqrt {\sin^2 \theta_0 - \sin^2 x} \, {\rm d}x =
-\frac{\cos x \sqrt{\cos 2x - \cos 2\theta_0}}{2 \sqrt{2}}
+\frac{\cos 2\theta_0+1}{4} \ln \left( \sqrt{2} \cos x + \sqrt{\cos 2x - \cos 2\theta_0} \right)
\end{equation}
For the special case of $x=0$, the integral becomes
\begin{equation}
-\frac{\sqrt{1 - \cos 2\theta_0}}{2 \sqrt{2}}
+\frac{\cos 2\theta_0+1}{4} \ln \left( \sqrt{2} + \sqrt{1 - \cos 2\theta_0} \right)
\end{equation}
while for the special case $x=\theta_0$, it reduces to
\begin{equation}
\frac{\cos 2\theta_0+1}{4} \ln \left( \sqrt{2} \cos \theta_0 \right)
\end{equation}
Using these equations, we can compute the required integrals:
\begin{eqnarray}
I_1 = \int_0^{\theta_{\rm in}} \sin \theta \sqrt {\sin^2 \theta_{\rm out} - \sin^2 \theta} \, {\rm d}\theta & = &
\frac{\sqrt{1 - \cos 2\theta_{\rm out}} - 
               \cos \theta_{\rm in} \sqrt{\cos 2\theta_{\rm in} - \cos 2\theta_{\rm out}}}
              {2 \sqrt{2}} \nonumber \\
& + & \frac{\cos 2\theta_{\rm out}+1}{4} \ln 
\left( 
\frac{\sqrt{2} \cos \theta_{\rm in} + \sqrt{\cos 2\theta_{\rm in} - \cos 2\theta_{\rm out}}}
        {\sqrt{2} + \sqrt{1 - \cos 2\theta_{\rm out}}} 
\right)
\end{eqnarray}
\begin{eqnarray}
I_2 = \int_0^{\theta_{\rm in}} \sin \theta \sqrt {\sin^2 \theta_{\rm in} - \sin^2 \theta} \, {\rm d}\theta & = 
&\frac{\sqrt{1 - \cos 2\theta_{\rm in}}}{2 \sqrt{2}} + 
\frac{\cos 2\theta_{\rm in}+1}{4} \ln 
\left( 
\frac{\sqrt{2} \cos \theta_{\rm in}}{\sqrt{2} + \sqrt{1 - \cos 2\theta_{\rm in}}} 
\right)
\end{eqnarray}
\begin{eqnarray}
I_3 = \int_{\theta_{\rm in}}^{\theta_{\rm out}} \sin \theta \sqrt {\sin^2 \theta_{\rm out} - \sin^2 \theta} \, {\rm d}\theta & = &
\frac{\cos \theta_{\rm in} \sqrt{\cos 2\theta_{\rm in} - \cos 2\theta_{\rm out}}}{2 \sqrt{2}} \nonumber \\
& + &\frac{\cos 2\theta_{\rm out}+1}{4}
\ln \left( 
\frac{\sqrt{2} \cos \theta_{\rm out}}
{ \sqrt{2} \cos \theta_{\rm in} + \sqrt{\cos 2\theta_{\rm in} - \cos 2\theta_{\rm out}} }
\right)
\end{eqnarray}
We thus obtain
\begin{eqnarray}
\frac{1}{2\pi f_0} & = & I_1 - I_2 + I_3 \nonumber \\
& = & \frac{ \sqrt{1 - \cos 2\theta_{\rm out}} - \sqrt{1 - \cos 2\theta_{\rm in}} }{ 2 \sqrt{2} } \nonumber \\
& + & \frac{\cos 2\theta_{\rm out}+1}{4} \ln \left( 
\frac{\sqrt{2} \cos \theta_{\rm out}} {\sqrt{2} + \sqrt{1 - \cos 2\theta_{\rm out}}} \right) -
\frac{\cos 2\theta_{\rm in}+1}{4} \ln \left( 
\frac{\sqrt{2} \cos \theta_{\rm in}}{\sqrt{2} + \sqrt{1 - \cos 2\theta_{\rm in}}} \right)
\end{eqnarray}
The ratio in $\frac{1}{f_0}$ between the small angle approximation and the correct spherical
computation is shown in Fig. \ref{fig:shellfunction} as function of $\theta_{\rm out}$ for
$\theta_{\rm in} = \frac{2}{3}\theta_{\rm out}$.

%%%%%%%%%%%%%%%%%%%%%%%%%%%%%%%%%%%%%%%%%%%%%%%%%%%
\begin{figure}
  \center
  \epsfig{figure=ShellFunctionApprox.eps, height=8cm}
 \caption{Ratio in $\frac{1}{f_0}$ between the small angle approximation and the correct 
spherical computation as function of $\theta_{\rm out}$ for
$\theta_{\rm in} = \frac{2}{3}\theta_{\rm out}$.}
 \label{fig:shellfunction}
\end{figure}
%%%%%%%%%%%%%%%%%%%%%%%%%%%%%%%%%%%%%%%%%%%%%%%%%%%


%%%%%%%%%%%%%%%%%%%%%%%%%%%%%%%%%%%%%%%%%%%%%%%%%%%
\subsection{Elliptical disk}

The effective radius $\theta_0$ of the ellipse on the sphere is given by
\begin{equation}
\theta_0 = \frac{1}{\sqrt{\frac{\cos^2(\phi-\phi_0)}{a^2} + \frac{\sin^2(\phi-\phi_0)}{b^2}}}
\end{equation}
where
$\phi_0$ is the position angle of the ellipse,
$\phi$ is the position angle of the position of interest on the sky with respect to the ellipse centre,
$a$ is the semi-major axis of the ellipse, and
$b$ is the semi-major axis.
This formula can be rewritten as
\begin{equation}
\theta_0 = \frac{ab}{\sqrt{b^2 \cos^2(\phi-\phi_0) + a^2 \sin^2(\phi-\phi_0)}}
\end{equation}
which is the formula implemented in {\tt GModelSpatialEllipticalDisk::eval()}.

The solid angle subtended by the ellipse is given by 
\begin{equation}
\Omega = 2 \pi \sqrt{(1-\cos a) (1-\cos b)}
\end{equation}
which is the formula implemented in {\tt GModelSpatialEllipticalDisk::update()}.
{\bf WARNING: This formula has not yet been verified.}


%%%%%%%%%%%%%%%%%%%%%%%%%%%%%%%%%%%%%%%%%%%%%%%%%%%
% Spectral models
%%%%%%%%%%%%%%%%%%%%%%%%%%%%%%%%%%%%%%%%%%%%%%%%%%%
\section{Spectral models}

%%%%%%%%%%%%%%%%%%%%%%%%%%%%%%%%%%%%%%%%%%%%%%%%%%%
\subsection{PowerLaw}

The power law spectral model is defined by
\begin{equation}
I(E) = k \left( \frac{E}{p} \right)^{\gamma}
\end{equation}
where
\begin{itemize}
\item $k$ is the normalization of the power law (units: ph cm$^{-2}$ s$^{-1}$ MeV$^{-1}$),
\item $p$ is the pivot energy (units: MeV), and
\item $\gamma$ is the spectral index (which is usually negative).
\end{itemize}

Each of the 3 parameters is factorised into a scaling factor and a value, i.e.
$k=k_s k_v$, $p=p_s p_v$, and $\gamma = \gamma_s \gamma_v$.
The {\tt GModelSpectralPlaw::eval\_gradients} returns the gradients with
respect to the parameter value of the factorisation.
Note that for any parameter $a=a_s a_v$:
\begin{equation}
\frac{\delta I}{\delta a_v} = \frac{\delta I}{\delta a} \frac{\delta a}{\delta a_v} =
  \frac{\delta I}{\delta a} a_s
\end{equation}
The parameter value gradients for the power law are given by
\begin{eqnarray}
\frac{\delta I}{\delta k_v} & = & 
  k_s \left( \frac{E}{p} \right)^{\gamma} = \frac{I(E)}{k_v} \\
\frac{\delta I}{\delta p_v} & = & 
  -\frac{\gamma}{p_v} \, k \left( \frac{E}{p} \right)^{\gamma} =
  -\frac{\gamma}{p_v} \, I(E) \\
\frac{\delta I}{\delta \gamma_v} & = &
  \gamma_s \ln \left( \frac{E}{p} \right) k \left( \frac{E}{p} \right)^{\gamma} =
  \gamma_s \ln \left( \frac{E}{p} \right) I(E)
\end{eqnarray}


%%%%%%%%%%%%%%%%%%%%%%%%%%%%%%%%%%%%%%%%%%%%%%%%%%%
\subsection{PowerLawPhotonFlux}

This flavour of the power law spectral model uses the integral photon flux $f$ within the
energy range $E_{\rm min}$ and $E_{\rm max}$ as free parameter instead of
the normalization $k$.
The integral flux $f$ is given by
\begin{equation}
f = \int_{E_{\rm min}}^{E_{\rm max}} I(E) {\rm d}E
\label{eq:fluxint}
\end{equation}
The power law model is then defined by
\begin{equation}
I(E) = \tilde{k} \, E^{\gamma}
\end{equation}
where
\begin{equation}
   \tilde{k} = \left \{
   \begin{array}{l l}
      \displaystyle
      \frac{f}{\ln E_{\rm max} - \ln E_{\rm min}} 
        & \mbox{if $\gamma = -1$} \\
     \\
     \displaystyle
      \frac{f (1+\gamma)}{E_{\rm max}^{\gamma+1}-E_{\rm min}^{\gamma+1}} 
        & \mbox{else}
   \end{array}
   \right .
\end{equation}
is obtained from Eq.~(\ref{eq:fluxint}).

Each of the 4 parameters is factorised into a scaling factor and a value, e.g.
$f=f_s f_v$ and $\gamma=\gamma_s \gamma_v$.
It is assumed that $E_{\rm min}$ and $E_{\rm max}$ are fixed parameters,
and {\tt GModelSpectralPlawPhotonFlux::eval\_gradients} returns valid gradients only for 
$f_v$ and $\gamma_v$.

The flux value gradient $f_v$ is given by
\begin{equation}
\frac{\delta I}{\delta f_v} = 
  \frac{\delta I}{\delta f} \frac{\delta f}{\delta f_v} = 
  \frac{\delta I}{\delta f} f_s = 
  \frac{\delta \tilde{k}}{\delta f} \, E^{\gamma} f_s = 
  \frac{\tilde{k}}{f} \, E^{\gamma} f_s = 
  \frac{I(E)}{f} f_s =
  \frac{I(E)}{f_v}
\end{equation}

The index value gradient $\gamma_v$ is given by
\begin{equation}
\frac{\delta I}{\delta \gamma_v} =
  \frac{\delta I}{\delta \gamma} \frac{\delta \gamma}{\delta \gamma_v} = 
  \frac{\delta I}{\delta \gamma} \gamma_s = 
  \left( \frac{\delta \tilde{k}}{\delta \gamma} \, E^{\gamma} + \tilde{k} \, E^{\gamma} \ln E \right)  \gamma_s =
  \left( \frac{1}{\tilde{k}} \frac{\delta \tilde{k}}{\delta \gamma} + \ln E \right) \tilde{k} E^{\gamma} \gamma_s =
  \left( \frac{1}{\tilde{k}} \frac{\delta \tilde{k}}{\delta \gamma} + \ln E \right) I(E) \gamma_s
\end{equation}
where
\begin{equation}
   \frac{\delta \tilde{k}}{\delta \gamma} = \left \{
   \begin{array}{l l}
     \displaystyle
     0 & \mbox{if $\gamma = -1$} \\
     \\
     \\
     \displaystyle
      \frac{f \left( E_{\rm max}^{\gamma+1}-E_{\rm min}^{\gamma+1} \right) -
              f (1+\gamma) \left( E_{\rm max}^{\gamma+1} \ln E_{\rm max} - 
                                      E_{\rm min}^{\gamma+1} \ln E_{\rm min} \right)}
              {\left( E_{\rm max}^{\gamma+1}-E_{\rm min}^{\gamma+1} \right)^2}
         & \mbox{else}
   \end{array}
   \right .
\end{equation}
Note that for $\gamma \ne -1$
\begin{equation}
\frac{1}{\tilde{k}} \frac{\delta \tilde{k}}{\delta \gamma} =
  \frac{1}{1+\gamma} -
  \frac{\left( E_{\rm max}^{\gamma+1} \ln E_{\rm max} - 
                    E_{\rm min}^{\gamma+1} \ln E_{\rm min} \right)}
          {\left( E_{\rm max}^{\gamma+1}-E_{\rm min}^{\gamma+1} \right)}
\end{equation}


%%%%%%%%%%%%%%%%%%%%%%%%%%%%%%%%%%%%%%%%%%%%%%%%%%%
\subsection{PowerLawEnergyFlux}

This flavour of the power law spectral model uses the integral energy flux $F$ within the
energy range $E_{\rm min}$ and $E_{\rm max}$ as free parameter instead of
the normalization $k$.
The integral energy flux $F$ is given by
\begin{equation}
F = \int_{E_{\rm min}}^{E_{\rm max}} I(E) {\rm d}E
\label{eq:fluxint}
\end{equation}
The power law model is then defined by
\begin{equation}
I(E) = \tilde{k} \, E^{\gamma}
\end{equation}
where
\begin{equation}
   \tilde{k} = \left \{
   \begin{array}{l l}
      \displaystyle
      \frac{F}{\ln E_{\rm max} - \ln E_{\rm min}} 
        & \mbox{if $\gamma = -2$} \\
     \\
     \displaystyle
      \frac{F (2+\gamma)}{E_{\rm max}^{\gamma+2}-E_{\rm min}^{\gamma+2}} 
        & \mbox{else}
   \end{array}
   \right .
\end{equation}
is obtained from Eq.~(\ref{eq:fluxenergy}).

Each of the 4 parameters is factorised into a scaling factor and a value, e.g.
$F=F_s F_v$ and $\gamma=\gamma_s \gamma_v$.
It is assumed that $E_{\rm min}$ and $E_{\rm max}$ are fixed parameters,
and {\tt GModelSpectralPlawEnergyFlux::eval\_gradients} returns valid gradients only for 
$F_v$ and $\gamma_v$.

The flux value gradient $f_v$ is given by
\begin{equation}
\frac{\delta I}{\delta F_v} = 
  \frac{\delta I}{\delta F} \frac{\delta F}{\delta F_v} = 
  \frac{\delta I}{\delta F} F_s = 
  \frac{\delta \tilde{k}}{\delta F} \, E^{\gamma} F_s = 
  \frac{\tilde{k}}{F} \, E^{\gamma} F_s = 
  \frac{I(E)}{F} F_s =
  \frac{I(E)}{F_v}
\end{equation}

The index value gradient $\gamma_v$ is given by
\begin{equation}
\frac{\delta I}{\delta \gamma_v} =
  \frac{\delta I}{\delta \gamma} \frac{\delta \gamma}{\delta \gamma_v} = 
  \frac{\delta I}{\delta \gamma} \gamma_s = 
  \left( \frac{\delta \tilde{k}}{\delta \gamma} \, E^{\gamma} + \tilde{k} \, E^{\gamma} \ln E \right)  \gamma_s =
  \left( \frac{1}{\tilde{k}} \frac{\delta \tilde{k}}{\delta \gamma} + \ln E \right) \tilde{k} E^{\gamma} \gamma_s =
  \left( \frac{1}{\tilde{k}} \frac{\delta \tilde{k}}{\delta \gamma} + \ln E \right) I(E) \gamma_s
\end{equation}
where
\begin{equation}
   \frac{\delta \tilde{k}}{\delta \gamma} = \left \{
   \begin{array}{l l}
     \displaystyle
     0 & \mbox{if $\gamma = -2$} \\
     \\
     \\
     \displaystyle
      \frac{F \left( E_{\rm max}^{\gamma+2}-E_{\rm min}^{\gamma+2} \right) -
              F (1+\gamma) \left( E_{\rm max}^{\gamma+2} \ln E_{\rm max} - 
                                      E_{\rm min}^{\gamma+2} \ln E_{\rm min} \right)}
              {\left( E_{\rm max}^{\gamma+2}-E_{\rm min}^{\gamma+2} \right)^2}
         & \mbox{else}
   \end{array}
   \right .
\end{equation}
Note that for $\gamma \ne -2$
\begin{equation}
\frac{1}{\tilde{k}} \frac{\delta \tilde{k}}{\delta \gamma} =
  \frac{1}{2+\gamma} -
  \frac{\left( E_{\rm max}^{\gamma+2} \ln E_{\rm max} - 
                    E_{\rm min}^{\gamma+2} \ln E_{\rm min} \right)}
          {\left( E_{\rm max}^{\gamma+2}-E_{\rm min}^{\gamma+2} \right)}
\end{equation}

\end{document} 
