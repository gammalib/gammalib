%---------------------------------------------------------------------------- 
%                            GammaLib User Manual
%
% Version .........: 0.0.0
% Last modification: 26 October 2006
%----------------------------------------------------------------------------

%%%%%%%%%%%%%%%%%%%%%%%%%%%%%%%%%%%%%%%%%%%%%%%%%%%%%%%%%%%%%%%%%%%%%%%%%%%%%
% Definitions for manual package
%%%%%%%%%%%%%%%%%%%%%%%%%%%%%%%%%%%%%%%%%%%%%%%%%%%%%%%%%%%%%%%%%%%%%%%%%%%%%
\newcommand{\task}{\mbox{GammaLib}}
\newcommand{\this}{\mbox{\tt \task}}
\newcommand{\version}{\mbox{0.0.0}}
\newcommand{\calendar}{\mbox{26 October 2006}}


%%%%%%%%%%%%%%%%%%%%%%%%%%%%%%%%%%%%%%%%%%%%%%%%%%%%%%%%%%%%%%%%%%%%%%%%%%%%%
% Document definition
%%%%%%%%%%%%%%%%%%%%%%%%%%%%%%%%%%%%%%%%%%%%%%%%%%%%%%%%%%%%%%%%%%%%%%%%%%%%%
\documentclass{article}[12pt,a4]
\usepackage{epsfig}
\usepackage{manual}


%%%%%%%%%%%%%%%%%%%%%%%%%%%%%%%%%%%%%%%%%%%%%%%%%%%%%%%%%%%%%%%%%%%%%%%%%%%%%
% Begin of document body
%%%%%%%%%%%%%%%%%%%%%%%%%%%%%%%%%%%%%%%%%%%%%%%%%%%%%%%%%%%%%%%%%%%%%%%%%%%%%
\begin{document}
\frontpage


%%%%%%%%%%%%%%%%%%%%%%%%%%%%%%%%%%%%%%%%%%%%%%%%%%%%%%%%%%%%%%%%%%%%%%%%%%%%%
% Introduction
%%%%%%%%%%%%%%%%%%%%%%%%%%%%%%%%%%%%%%%%%%%%%%%%%%%%%%%%%%%%%%%%%%%%%%%%%%%%%
\section{Introduction}

%%%%%%%%%%%%%%%%%%%%%%%%%%%%%%%%%%%%%%%%%%%%%%%%%%%%%%%%%%%%%%%%%%%%%%%%%%%%%
\subsection{Motivation}

Again another library?
Indeed, 

%%%%%%%%%%%%%%%%%%%%%%%%%%%%%%%%%%%%%%%%%%%%%%%%%%%%%%%%%%%%%%%%%%%%%%%%%%%%%
\subsection{Philisophy}

%%%%%%%%%%%%%%%%%%%%%%%%%%%%%%%%%%%%%%%%%%%%%%%%%%%%%%%%%%%%%%%%%%%%%%%%%%%%%
\subsection{Implementation}


%%%%%%%%%%%%%%%%%%%%%%%%%%%%%%%%%%%%%%%%%%%%%%%%%%%%%%%%%%%%%%%%%%%%%%%%%%%%%
% GammaLib objects
%%%%%%%%%%%%%%%%%%%%%%%%%%%%%%%%%%%%%%%%%%%%%%%%%%%%%%%%%%%%%%%%%%%%%%%%%%%%%
\section{\this\ objects}

%%%%%%%%%%%%%%%%%%%%%%%%%%%%%%%%%%%%%%%%%%%%%%%%%%%%%%%%%%%%%%%%%%%%%%%%%%%%%
\subsection{Vectors and matrices}

%----------------------------------------------------------------------------
\subsubsection{General}

Most of the calculations that are needed in data analysis can be 
expressed as vector and matrix operations.
For this reason, \this\ provides a vector and matrix package that is 
implemeted as the {\tt GVector} and the {\tt GMatrix} classes.


%----------------------------------------------------------------------------
\subsubsection{Matrix types}

The matrix base class is {\tt GMatrix} which stores all elements of 
the matrix (including zeros) in a full array.
The full array has the dimensions {\tt rows} $\times$ {\tt cols}, 
matrix elements are stored in column major format.

{\tt GSymMatrix} symmetric matrix.

{\tt GSparseMatrix} sparse matrix.

{\tt GSymSparseMatrix} symmetric sparse matrix.

%----------------------------------------------------------------------------
\subsubsection{Matrix arithmetics}

{\tt GMatrix} and its derived objects can be added, subtracted, 
scaled and negated, very much as other data types.
The following code fragment illustrates the usage:
\begin{verbatim}
 // Matrix allocation
 GMatrix a(2,3);
 GMatrix b(3,2);
 GMatrix c(2,2);
 GMatrix d(2,2);
 
 // Element assignment
 a(0,0) = a(1,1) = a(1,2) = 1.0;
 b(0,0) = b(1,1) = b(2,1) = 1.0;
 d(0,0) = d(1,1) = 1.0;
 
 // Arithmetics
 c = a * b;           // Matrix multiplication
 d = c;               // Assignment
 c = c * 2.0;         // Multiplication of all matrix elements by 2.0
 a = a + 1.0;         // Addition of 1.0 to all matrix elements
 d = d + d - c;       // Matrix addition and subtraction
 c = -d;              // Negation of all matrix elements
 a *= 5.0;            // Multiplication of all matrix elements by 5.0
\end{verbatim}
Matrix allocation requires {\bf always} the specification of the number 
of rows and columns of the matrix.
This means that there is no such thing as a matrix object with zero 
rows or columns.
All matrix elements are initialised to 0.0 by the matrix allocation.

Matrix elements are accessed by the {\tt matrix(row,col)} function,
where {\tt row} and {\tt col} start from 0 for the first row or column 
and run up to the number of rows or columns minus 1.

Matrix arithmetics comprise binary operators {\tt C = A operator B} and
unary operators {\tt C operator A}. {\tt A} and {\tt B} may be either
{\tt GMatrix} (or derived) objects or {\tt double} values. In case of
{\tt double} values the operation is performed on each matrix element,
i.e.
\begin{verbatim}
 c = c * 2.0;
 d = d / 3.0;
\end{verbatim}
multiply each matrix element of {\tt c} by 2.0 and divide each matrix 
element of {\tt d} by 3.0.
These operations can be writted also more compactly using unary 
operators:
\begin{verbatim}
 c *= 2.0;
 d /= 3.0;
\end{verbatim}
While the addition, multiplication

%----------------------------------------------------------------------------
\subsubsection{Matrix functions}


%----------------------------------------------------------------------------
\subsubsection{Matrix compression}

The {\tt GSymMatrix} Cholesky decomposition, solver and inversion
routines may also be applied to matrices that contain rows or
columns that are filled by zeros.
In this case the functions provide the option to (logically)
compress the matrices by skipping the zero rows and columns during
the calculation.


%%%%%%%%%%%%%%%%%%%%%%%%%%%%%%%%%%%%%%%%%%%%%%%%%%%%%%%%%%%%%%%%%%%%%%%%%%%%%
% Code reference
%%%%%%%%%%%%%%%%%%%%%%%%%%%%%%%%%%%%%%%%%%%%%%%%%%%%%%%%%%%%%%%%%%%%%%%%%%%%%
\clearpage
\section{Code reference}

%%%%%%%%%%%%%%%%%%%%%%%%%%%%%%%%%%%%%%%%%%%%%%%%%%%%%%%%%%%%%%%%%%%%%%%%%%%%%
\subsection{{\tt GVector}}

%%%%%%%%%%%%%%%%%%%%%%%%%%%%%%%%%%%%%%%%%%%%%%%%%%%%%%%%%%%%%%%%%%%%%%%%%%%%%
\subsection{{\tt GMatrix}}

\end{document} 
